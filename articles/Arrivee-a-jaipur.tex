\section{Arrivée à Jaipur}

6 fév. 2008

\begin{multicols}{2}

Que de choses à raconter...

Premièrement faisons un point sur ce petit début de voyage, nous sommes donc partis de Delhi vers Agra en train, c'est là que nous avons pu voir le Fort rouge et le Taj Mahal, pas gigantesque, mais magnifique, en marbre blanc. Après avoir passé deux jours à Agra nous voulions remonter un peu sur Mathura pour voir les ablussions dans le fleuve Yamuna tôt le matin (les croyants descendent les gaths (escaliers en pierre tombant dans le fleuve) pour se baigner dans le fleuve et ainsi se purifier.

La gare de bus ne nous donnant aucune confiance car il y a trop de personnes ici prêtes à faire monter le touriste dans un mauvais bus pour qu'il soit ensuite obligé de payer un hôtel hors de prix ne sachant pas où il est. Nous avons décidé de prendre le train, en general class (la plus basse des six classes possibles ici, cela se traduit par un entassement de personnes dans un seul wagon, aucune assurance qu'il y ai assez de place, certains sont assis dans les range-bagages au dessus des têtes (je précise que le compartiment bagages n'est pas plus grand que dans un train français en seconde classe, sauf que c'est du grillage).

Malheureusement nous n'étions pas à la bonne saison et seuls les gens de Mathura se baignaient, loins des centaines de personnes que nous nous attendions à voir. Par contre nous avons vu des singes, des singes de partout, surtout tôt le matin. En Inde les singes sont quasiment comme les pigeons à Paris, seulement les Indiens respectent tous les animaux car ils croient en la réincarnation, ne voulant pas tuer un ancêtre potentiel, ils se contentent de les repousser quand ils le peuvent.

Mais Mathura est aussi connue pour être la ville de naissence Krishna, principale incarnation de Vishnu et qui est très vénéré par les Hindous (c'est de là que vient la secte Hare Krishna). Il y avait donc un grand nombre de temples dans cette ville, ainsi qu'une atmosphère religieuse très pesante, nous ne nous sentions pas à notre place c'est pourquoi nous n'y avons passé qu'une nuit.

Le lendemain donc, objectif être à Jaipur le soir, il y avait 300Km à parcourir avec escale à Agra. Retour à Agra en "gen class", puis nous voulions tenter le bus pour Jaipur (l'astuce est que ce bus la ne part pas de la gare routière, nous avions donc plus confiance). Je passe quelques coups de fil à Jaipur pour savoir où dormir, tout est plein, plus le temps, tant pis on part on verra là bas. Seulement les transports ici, on sait quand ça part mais... on sait pas vraiment quand ça arrive, tant mieux on verra bien, si les horaires sont respectés on doit arriver à 22h.

Et la, coup de bol, d'une les horaires sont respectés, et de deux sur la fin du trajet un des passagers nous avait entendu parler francais, il vient nous parler dans cette langue, on lui dit qu'on ne sait pas où on va en sortant du bus, qu'on ne sait pas où s'arrête le bus, "vous voyagez en routard, c'est bien". Il nous indique où aller pour dormir pour pas cher, à deux pas, même pas besoin de prendre un tuk tuk (rickshaw, taxi à trois roues comme sur une des photos d'un post précédent).

Aujourd hui nous avons passé la journée tranquillement dans la ville en se baladant dans les bazaars, qui sont leurs marchés où l'ont peu trouver de tout, des pneus à côté d'une bijouterie, d'un magasin d'étoffes ou d'un vendeur de légumes. Ensuite farniente dans un parc public puis nous voici devant une connection internet des plus lentes qu'il puisse être. nous ne pouvons pas vous mettre de photos aujourd'hui mais promis, pour la prochaine fois.

\end{multicols}

\bigskip
\textbf{\textsc{Commentaires}}

\medskip
Joseph Roubignolles a écrit le 06 fév. 2008 :
\begin{displayquote}
J'aime l'esprit "on tente de planifier tout en sachant qu'on fera autre chose". Je dirais même plus, j'adore ! Que l'esprit du routard veille sur vous et vous guide ...
Bon trip !!
Poutine Spirit
\end{displayquote}

\medskip
Peggy a écrit le 06 fév. 2008 :
\begin{displayquote}
Et je vous ai manqué sur Internet... enfin... l'essentiel c'est que vous avanciez a bon rythme, c'est à dire: le votre.
\end{displayquote}

\medskip
Titou a écrit le 06 fév. 2008 :
\begin{displayquote}
Ro la classe ! J'avoue vous assurez de vous la jouer "guide du routard" qui ne sait pas ou il va ! Franchement respect car je ne sais pas si je pourrais le faire de moi meme ... faudrait que quelqu'un me pousse .. comme un ti dud par exemple :D Ca sera pour le prochain coup ! En tout cas je suis super content de voir que vous en profitez a fond et surtout continuez de vous perdre mais revenez quand meme a la fin ! A bientot les aventuriers et merci de nous faire voyager !
\end{displayquote}

\medskip
Poun's a écrit le 07 fév. 2008 :
\begin{displayquote}
Salut  les 2 explorateurs, merci pour vos nouvelles et votre carte de voyage, mais il est vrai que la seule chose que l'on peut prévoir, c'est que l'on ne peut rien prévoir. Faites Gaaafffe aux arnaaaaques, vous êtes habitués, mais y'a peut-être pire. et plus malins.
Vous pouvez nous ramener une vache sacrée, un singe et 1 l'd'eau du Gange en souvenir?
A bientôt, bon trip....
\end{displayquote}

\medskip
Les vadrouilleurs a écrit le 08 fév. 2008 :
\begin{displayquote}
Merci beaucoup pour vos encouragements, vos commentaires nous font toujours tres plaisir. On voit que certains suivent notre voyage de tres pres.
Continuez comme ca, daniwad.
\end{displayquote}

\medskip
Tatid a écrit le 21 fév. 2008 :
\begin{displayquote}
"les Indiens respectent tous les animaux car ils croient en la reincarnation, ne voulant pas tuer un ancetre potentiel, ils se contentent de les repousser quand ils le peuvent." <= je trouve cette phrase EXCEPTIONNELLE ! Ne voulant pas tuer un ancêtre, trop fort... Forcément avec leur croyance en la réincarnation !
C'est toujours un plaisir de vous lire, bon courage pour la suite !
\end{displayquote}


