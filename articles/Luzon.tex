\section{Luzon}

Date: 30/12/2011

\begin{multicols}{2}

Coucou, me revoila...

Je viens avec un peu de retard donner des nouvelles, pour la suite de mon passage aux Philippines. L'article d'aujourd'hui ne comportera que très peu de photos car les coins où je suis allé dans la seconde partie de ma boucle sont beaucoup moins touristiques et les paysages ne sont pas forcément aussi magnifiques que ceux de mon précédent article (en même temps la barre était très haute, j'aurai du mal à retrouver des si belles photos que celles que j'ai faites à Palawan..).

En fait ma boucle au nord de Luzon a surtout été l'occasion de faire des rencontres, et ça c'est difficile à raconter, je ne vais donc que vous faire un court résumé de ma semaine ici.. en attendant Bali..

J'ai donc quitt' Palawan en avion pour retourner à Manille, sale, bondée, bouchée, polluée.. j'y suis resté le temps de trouver un bus qui parte vers le nord. Mon plan au départ était d'aller à Vigan au nord de l'île de Luzon (iîle où est Manille) mais comme il fallait attendre le bus plus de trois heure, je me suis décidé à faire un truc que j'aime bien : à savoir monter dans le premier bus au départ sans savoir sa destination. C'est parti pour 6 heures de trajet direction San Fernando (qui se trouve être sur la route de Vigan). De là j'ai posé mon sac en ville et durant trois jours je me suis ballade aux alentours, j'y ai fait des rencontres sympas telles que ce couple de tailleurs.

TODO : Couple de tailleurs

Puis au hazard d'un bus je me suis arrêté pour jouer au billard avec des jeunes, il se trouve que l'endroit appartient à l'un d'eux et son père a la concession de voiture d'en face.. et c'est parti pour boire des verres toute la soirée.

%<img src="http://etienne.croclemonde.org/public/Philippines/DSCF2278.jpg" />

Le lendemain on se revoit, ils me proposent alors d'aller à l'arene de combats de coqs pour l'après midi. Je suis très clairement contre le principe des combats de coqs, mais en même temps tiraillé par l'envie de voir cela de mes yeux car c'est ici une vrai culture, et je pense qu'il est important de l'avoir vu au moins une fois. Bon pour le coup je ne mettrait pas ici de photos des combats pour deux raisons : la première est que je ne veux pas montrer ça sur ce site, et la seconde c'est tout simplement parce qu il faisait sombre et du coup ça rend rien.

Voici quand meme l'arene.

%<img src="http://etienne.croclemonde.org/public/Philippines/DSCF2287.jpg" />

Les jours suivants je suis allé à Baguio dans la cordillière. De part son positionnement dans les montagnes Baguio profite de temperatures beaucoup moins élevées que dans le reste du pays. Ces températures fraîches (même pour moi) étant très recherchées par les Philipinos, Baguio devient une sorte de ville de vacances. J'ai donc essayé d'écumer les centres d'intérêts de cette ville. Et ben je vous assure que c'est pas jojo d'être touriste Philipino. Le principal point d'intérêt de la ville étant Mines View, un point de vue sur plusieurs vallées pas plus joli que ça et duquel on est censé deviner au loin des anciennes mines.. qu'on ne voit pas !! Et comme ça attire beaucoup de gens, l'endroit est en fait un amoncellement de boutiques de souvenirs comme j'ai rarement vu pour un lieu présentant si peu d'intérêt.

%<img src="http://etienne.croclemonde.org/public/Philippines/DSCF2298.jpg" />
%<img src="http://etienne.croclemonde.org/public/Philippines/DSCF2296.jpg" />

Ca y est c'est déjà la fin du voyage aux Philippines, retour à Manille en express et décollage à 6h du mat pour une longue escale à Kuala Lumpur dans la journée. J'en ai profité pour aller dans le quartier indien et me faire un petit resto de toute beauté (enfin l'assiète, parce que le resto..).

%<img src="http://etienne.croclemonde.org/public/Philippines/DSCF2306.jpg" />

Et devinez qui j'ai croisé.. Les petits lutins du père noel. Eh bien j'un scoop qui va faire parler dans les chaumières.. les lutins sont en fait des lutines..

%<img src="http://etienne.croclemonde.org/public/Philippines/DSCF2308.jpg" />

A bientôt pour la suite à Bali où je suis actuellement.. Toujours aussi belle île, mais de plus en plus de touriste au Sud.. au secours fuyons au nord.. Florian m'a rejoint hier, je suis allé le chercher à l'aéroport puis je lui ai montré les environs tout en précisant que le Sud n'est pas le vrai Bali. Puis repos pour encaisser le décallage horaire.

\end{multicols}

\bigskip
\textbf{\textsc{Commentaires}}

\medskip
Dova a écrit le 04 janv. 2012 :
\begin{displayquote}
Salut les mecs,
J'espère que vous allez bien.
Je profite de ce message pour vous souhaiter une très belle année avec tout pleins de bonnes choses pour vous... un travail agréable, des voyages aux destinations des plus authentiques et des rencontres chaleureuses ;)...
Grosses bises à vous deux et prenez soin de vous (ce que je ne doute pas; vous commencez très bien 2012!)
A très vite,
Dova
\end{displayquote}

\vfill

