\section{Journées zen}

11 fév. 2008

\begin{multicols}{2}

Après Jaipur, 2h30 de bus et nous voila à Ajmer, petite ville de 250.000 habitants si nos souvenirs sont bons. Le principal avantage du lieu est de se situer sur de nombreuses lignes de bus et d'être à seulement 15 km de Pushkar, ville de pèlerinage, très réputée pour sa beauté. Nous n'y avons donc passé qu'une nuit et qu'une journée. Toutefois nous en avons profité au maximum pour nous balader, prendre le soleil et admirer le paysage. Tout d'abord, visite du Temple Rouge Jaïn, qui nous a révélé une magnifique maquette de ville aux bateaux flottants dans les cieux, le tout recouvert de 500kg de feuillles d'or. Les murs étaient même sculptés en colonnes brillantes de vert, de rouge, de bleu.

Nous avons continué notre route jusqu'à un fort non loin. Il s'agit d'un musée contenant des tablettes de pierres gravées d'écritures et de scupltures à l'effigie des dieux. La cour intérieure était tellement verdoyante et paisible que nous y avons déposé nos sacs pour une halte détente, farniente et bouquinage. L'Inde, lorsqu'elle est calme, est propice au repos et à la reflexion. Le lac fut notre dernière visite dans cette ville. Le parc est accueillant et vivant, les quais sont en marbre avec des autels et l'eau calme reflète les montagnes environnantes. Premier changement radical de décor pour nous, ça ressemble plus au désert.

Le soir pour venir à Pushkar nous avons comme à notre habitude pris le bus. Enchantés dès l'arrivée par la petite bourgade de 15.000 habitants (ou plutôt ce qu'on arrive à en voir entre la station de bus et le premier hôtel où l'on entre), nous avons pu profiter de la suite du spectacle lors d'une balade dans le bazar, encore baigné de soleil. De retour le soir, on constate avec ravissement que notre hôtel porte bien son nom de Lake View. La terrasse (qui est le toît, je pense que vous l'aurez compris, comme partout en Inde) offre une vue imprenable sur le lac sacré de Pushkar, où viennent se baigner de nombreux fidèles. A cette heure, le soleil se couche, libérant des lueurs jaunes, oranges et violettes dans le ciel, l'eau est tranquille, les montagnes presque bleues et les lumières de la ville commencent à s'allumer. Nous restons là un long moment à admirer le soleil qui descend puis s'éteint. Le soir, repas autour du feu en compagnie d'un Français et des Indiens du restaurant sur fond de Bob Marley. Ce fut une journée tout à notre rythme, très agréable.

Aujourd'hui aussi, du coup, nous en avons profité juste en marchant, en faisant quelques emplettes et en profitant de la terrasse de l'hôtel. J'ai eu la chance, mais aussi l'audace de rentrer dans un temple, d'offrir des fleurs à Krishna et de lui faire une offrande de fleurs dans l'eau du lac. C'est une expérience qui je crois m'a beaucoup marquée, surtout moi qui suis si émotive, j'ai été très touchée.  Maintenant que la nuit est tombée, nous avons profité de cette ambiance si différente pour faire un saut jusqu'ici vous donner des nouvelles et vous montrer quelques photos.

\smallbreak
\hspace*{-0.65cm}
\includegraphics[width=5cm]{articles/Journees-zen/pushkar.jpg}
%La ville de Pushkar.
\smallbreak

\end{multicols}

\bigskip
\textbf{\textsc{Commentaires}}

 \medskip
Titou a écrit le 11 fév. 2008 :
\begin{displayquote}
Salut la jeunesse ! J'attendais avec impatience la suite de vos peripeties et comme d'habitude c'est avec une grande joie que je vous lis ! A vous lire ca fait rever et donne vraiment envie de repartir encore plus loin pour un choc de culture encore plus gros ! Je pense que vous plannez complet labas avec ce que vous voyez et je vous souhaite de continuer ca jusqu'au bout ! Merci pour ces news et a tres vite ! Biz a vous deux
\end{displayquote}

 \medskip
Lydie a écrit le 12 fév. 2008 :
\begin{displayquote}
De mon petit bureau monotone, je m'évade grâce à vous !	
Je suis subjuguée par les aventures que vous vivez !!!
Vous vivez une aventure qui restera à tout jamais gravé dans vos mémoires.
Ma petite Cécile, je dois t'avouer que je t'admire. Je suis bien incapable de partir ainsi à l'aventure (mon petit confort m'est si indispensable !).
Mais je dois également avouer que je t'envie, j'aimerais tenter et pouvoir être fier de l'avoir vécu !!!
Voir une culture si différente avec des besoins qui sont réels, doit donner une grande leçon d'humilité !
Vous reviendrez j'en suis sûre, avec une autre vision du monde et ça ne peut être que constructif !
Tu vas ainsi pouvoir relativiser des mésaventures que l'on peut connaitre dans notre palais occidental doré qu'est la France !
Mon esprit sécuritaire me pousse tout de même à vous dire de faire attention et de bien prendre soin de vous (sanitaire et sécuritaire)!
BRAVO à toi ma petite Cécile, ma petite routarde !
Je vous souhaite bonne route, profitez de chaque instant et remplissez votre mémoire de chaque moment, ce sont les seuls souvenirs qui resteront à jamais aussi fort !
Merci de nous faire partager ces moments
J'attends avec impatience vos prochains récits
Grosses bises à vous 2 et prenez soin de vous !
Lydie
PS : Je suis bien évidemment partante pour vous donner de l'aide afin que ces enfants aient accès à l'enseignement et autres ...
\end{displayquote}

 \medskip
Poun's a écrit le 12 fév. 2008 :
\begin{displayquote}
Salut les z'aventuriers. Ca fait plaisir d'avoir de bonnes nouvelles. Des images de plus en plus belles, continuez comme ça, profitez bien, mettez-en plein votre tête!
Coucou à Lydie en passant et message à toutes et tous :
On ne pose pas de questions techniques, on les laisse vivre leur trip et on leur tombera dessus lorsqu'ils seront rentrés! Ils en auront pour un bon moment à nous raconter.
A bientôt pour la suite...
\end{displayquote}

 \medskip
Tatid a écrit le 21 fév. 2008 :
\begin{displayquote}
Snif, j'arrive trop tard pour voir les photos, c'est de ma faute, j'avais qu'à lire le blog avant :-p
Rien que vos descritions nous laissent imaginer des paysages et monuments superbes !
\end{displayquote}


