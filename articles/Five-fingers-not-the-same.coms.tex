\bigskip
\textbf{\textsc{Commentaires}}

\medskip
Marie-Hélène a écrit le 3 fév. 2008 :
\begin{displayquote}
Super vos photos, vos commentaires, vous nous faîtes rêver! Nous vous accompagnons dans votre périple, et imaginons les bruits, les odeurs les couleurs et tout ce qui va avec la magie de ce pays!
Bonne continuation, continuez à faire de belles rencontres, et faites attention à vous.
Je vous embrasse !
\end{displayquote}

\medskip
Titou a écrit le 4 fév. 2008 :
\begin{displayquote}
Hey hey ! Ra ça donne envie d'y aller ! Arrêtez de nous narguer ça ne se fait pas ! Dire que dans 3 petites semaines c'est Belfort qui m'attend... Bon ok après 6 mois à l'autre bout du monde mais quand même ! Continuez de nous faire rêver et pour tout vous dire vous me donnez quelques idées de destinations pour mon ST50...
Eclatez vous comme des petits fous, faites vous plaisir et surtout faites gaffe à vous ! A plouche les amis !
\end{displayquote}

\medskip
Ewen a écrit le 4 fév. 2008 :
\begin{displayquote}
Salut Etienne ça va ça se passe	bien.
Bisous Moi et Nils pensons bien à toi.
\end{displayquote}

\medskip
Jean-yves a écrit le 4 fév. 2008 :
\begin{displayquote}
Nous suivons quasiment en temps réel cette plongée dans une société très différente. C'est fabuleux. Et puis en plus des images, il n'y peut être pas de son, mais il y du texte, du vrai, avec des petites pointes de poésie savoureuse. J'adore...
Il est vrai que La Savoureuse, ça vous connaît...
\end{displayquote}

\medskip
Gerien a écrit le 4 fév. 2008 :
\begin{displayquote}
Les gens ont vraiment l'air d'être super peace. C'est cool :)
Ça donne envie !!!!
\end{displayquote}

\medskip
Catherine a écrit le 6 fév. 2008 :
\begin{displayquote}
Merci pour tout ce que vous nous faites partager et à Cécile pour son mail.
J'adore la deuxième photo, nettement moins conventionnelle et plus parlante que celle que l'on voit habituellement. L'on aurait trop tendance à oublier que le Taj Mahal n'est pas le seul palais du lieu.
\end{displayquote}

\medskip
La vadrouilleuse a écrit le 10 fév. 2008 :
\begin{displayquote}
Il est vrai qu il y a d'autres curiosités à Agra que le Taj Mahal, dont le Fort Rouge qui m'a particulierement touchée. Mais pour info, on ne voit que le Taj sur la photo : il s'agit en effet autour des remparts, devant la porte sud, et au loin de deux annexes, le tout dans l'enceinte rose du Taj Mahal.
\end{displayquote}

\medskip
Tatid a écrit le 21 fév. 2008 :
\begin{displayquote}
Arf arf, plus de photos dispo, mais il me semble les avoir vues quand j'étais passé faire un tour sur votre blog =)
C'est fou qu'une mère confie son enfant comme ça à une inconnue, on verrait pas trop ça en France :-D ! C'est sympa que le temps soit toujours "au beau fixe" ! Profitez, ici à Paris, il pleut... ! :(
\end{displayquote}

\vfill
