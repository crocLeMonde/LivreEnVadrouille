\bigskip
\textbf{\textsc{Commentaires}}

 \medskip
Mamam titou a écrit le 14 avril 2008 :
\begin{displayquote}
ton pote t'a traité de salopard - je ne fais pas de  délation mais je l'ai bien entendu - comme d'hab les photos sont superbes - profites de ton séjour nautique tu es dans ton élément. bon courage et je pense que nous aurons l'occasion de nous revoir sur la région parisienne ....gros bisous de la famille titou
\end{displayquote}

 \medskip
Cécile a écrit le 14 avril 2008 :
\begin{displayquote}
magnifique! que dire de plus...
\end{displayquote}

 \medskip
Zan a écrit le 14 avril 2008 :
\begin{displayquote}
Yo!
Lors de mon st40, j'ai des potes qui sont partis à Bali et qui ont profité de Bali. C'était effectivement un peu des paysages dans le même genre sur leurs photos... Ma-Gnin-Fi-Que!
Profites veinard! :p
\end{displayquote}

 \medskip
Jaco a écrit le 15 avril 2008 :
\begin{displayquote}
Mouais ya beaucoup de vert quoi... C'est comme en Alsace, sauf que c'est pas du vert mais du gris, surtout dans le ciel.
Bonne continuation !
\end{displayquote}

 \medskip
Peggy a écrit le 16 avril 2008 :
\begin{displayquote}
Coucou!
et question alimentaire (même l'eau), aucun risque ?
Bizx
\end{displayquote}

 \medskip
Anick a écrit le 17 avril 2008 :
\begin{displayquote}
etienne...
\end{displayquote}

 \medskip
Monique et Alain a écrit le 17 avril 2008 :
\begin{displayquote}
nous comprenons ton enthousiasme à découvrir encore de nouvelles contrées, de nouvelles cultures! Soit prudent et bon vent pour ta traversée, évite les pirates du Détroit de Malacca!
\end{displayquote}

 \medskip
Etienne a écrit le 22 avril 2008 :
\begin{displayquote}
Hehe... et bien je vois aue les commentaires marchent bien.
Je suis actuellement a Kumai (prononcer coumaille) sur Borneo, il s'agit de l'enorme ile situee au nord de Bali et de Java.
En fait Borneo c'est mytique comme destination, c'est un repere historique de pirates.
J'ai evidement plein de choses a vous raconter, mais peu de temps sur internet pour tout dire, donc je vais vous faire patienter encore un peu, un autre jour, pour plus de nouvelles. Aujourd'hui l'objectif est plus de donner signe de vie apres nos 4 jours de traversee en mer.
Pour ta qustion Peggy je dirais que comme en Inde il faut (faudrait ?) faire tres attention a ce que l'ont mange, par contre mon point de vue a beaucoup change sur la question dans le sens ou je mange de tout, quitte a etre malade ensuite (ce qui n'est pas encore arrive) mais au moins j'en aurais profite. Reste l'eau, je continue a faire attention en mettant des pastilles desinfectantes dedans ou en achetant des bouteilles d'eau minerale (j'ai meme vu de la volvic!!!).
A bientot tout le monde pour un nouvel article
\end{displayquote}

 \medskip
Soeurette... a écrit le 22 avril 2008 :
\begin{displayquote}
He ben, ça fait du bien  d'avoir des nouvelles, je commencais à m'inquièter...!!! 
Profit bien
Bisou
Soeurette
\end{displayquote}

 \medskip
Gerien a écrit le 24 avril 2008 :
\begin{displayquote}
Et bé... ça donne envie...
Y'a encore quelques mois, je te connaissais pas une âme de routard comme ça etienne. :)
Moi le seul paysage que je peux "admirer", c'est le mont salbert par la fenêtre de mon bureau (et encore, quand il y a pas trop de nuages (pas comme les huit dernières semaines) )...
Profites bien de la traversée ;-)
\end{displayquote}

 \medskip
Etienne a écrit le 30 avril 2008 :
\begin{displayquote}
Je l'avais deja cette ame de routard, gerien, mais c'est vrai qu'a belfort c'est difficile de le voir, on trouve d'autres choses pour s'evader ;-) En fait ca fait tres longtemps que j'ai envie de voyager, et ce voyage n'est que le premier...
\end{displayquote}

