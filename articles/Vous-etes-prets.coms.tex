\bigskip
\textbf{\textsc{Commentaires}}

\medskip
Didier a écrit le 13 nov. 2010 :
\begin{displayquote}
Bon voyage mec! Ca va être du bon\dots
PS: et sois un peu moins négligeant avec ton blog ;-)
A++
Didier.
\end{displayquote}

\medskip
Nicoz et Emilie a écrit le 14 nov. 2010 :
\begin{displayquote}
Bon voyage !!
Profites en bien :)
Biz
Nico et Emilie.
\end{displayquote}

\medskip
Macolu a écrit le 14 nov. 2010 :
\begin{displayquote}
Veinard ! Profites-en bien !
\end{displayquote}

\medskip
Dodo a écrit le 14 nov. 2010 :
\begin{displayquote}
Un petit voyage qui s'annonce bien ! Profite bien !
Et que la Dud attitude t'accompagne
La bise.
\end{displayquote}

\medskip
Dovaline a écrit le 16 nov. 2010 :
\begin{displayquote}
Coucou Loulou,
Kiffes bien tes 4 semaines de vacances et fais nous rêver si peu par des commentaires et photos.
Gros Zoubis.
\end{displayquote}

\medskip
Sonia a écrit le 18 nov. 2010 :
\begin{displayquote}
Coucou Étienne
Une p'tit bonjour de "Paris" où il pleut toujours.
Il y a combien de décalage horaire?
Profites et fais tourner :)
@+
Sonia.
\end{displayquote}

\medskip
Etienne a écrit le 19 nov. 2010 :
\begin{displayquote}
Salut tout le monde, je suis aujourd'hui à Manakara et je pars pour Ambalavao ce soir. Le premier article sur ce blog risque d'attendre encore quelques jours mais il devrait être fort en récits et en photos car les paysages sont vraiment fabuleux. Pas de problèmes politiques en vue où je suis, n'écoutez pas les infos le problème semble en fait cantonné à une partie de la capitale et se règle surtout entre forces armées et normalement devrait se boucler autour d'une table dans les jour qui viennent. Je reste informé dans tous les cas pour éviter les endroits qui craignent.
@Sonia : L'heure de Mada est égale à l'heure de la France +2h, et comme on est vers le tropique le soleil se lève vers 5h30 et se couche vers 18h30.
\end{displayquote}

\vfill
