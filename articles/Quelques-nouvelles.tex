\section{Quelques nouvelles}

\par
\par
Date: 02/11/2014

\par
\par
%\begin{multicols}{2}

\par
\par
Bonsoir à vous tous, chers co-organisateurs en évènementiel !

\par
\par
Comme je vous l’avais promis, je vous envoie un peu plus d’informations sur Bali, et sur ce qui m’a pris beaucoup de temps et d’énergie l’année dernière. C’est après plusieurs relance, dont la dernière aujourd’hui ;) que je m’y met, car ce texte me remémore pleins de choses comme vous allez pouvoir le lire, il est donc très chargé en émotions pour moi.

\par
\par
Le texte que je vous envoie a été rédigé juste avant que je revienne en France en décembre, j’ai hésité à le mettre à jour avant de vous l'envoyer mais sont principal intérêt il me semble est de transmettre le point de vue que j’avais à ce moment. En le relisant après la semaine que l’on a partagée, j’aurais forcément pleins de commentaires à faire au niveau de la gestion de projet, vous en aurez sans doute aussi.

\par
\par
Les quelques éléments que j’ai ajoutés depuis décembre sont en italique. J’ai ajouter aussi à la fin quelques liens et des photos pour vous illustrer le récit.

\par
\par
Avant de commencer, je vais déjà vous resituer les événements : Bali est l’une des 17 500 îles de l’Indonésie, dans le top 20 en terme de taille. L’Indonésie est le plus grand pays musulman du monde, toutes les îles sont donc à majorité de cette religion. Toutes les îles ? Toutes, sauf une qui résiste encore et toujours : Bali est à majorité (94\%) hindoue, ce qui en fait une exception dans le pays et un attrait touristique énorme, grâce à ses paysages bien sur (soleil, rizières, surf, plongées, …), mais surtout à sa population qui croit dans le Kharma et qui a des cérémonies religieuses magnifiques, des gens dont le sourire nous fait nous poser beaucoup de questions…

\par
\par
Voici enfin un peu plus de news sur ce qu’il s’est passé ici ces derniers temps, ayant pas mal bougé depuis, les conditions ont rarement été réunies pour prendre le temps d’écrire, et la tête n’y était pas… Je m’excuse par avance du ton un peu impersonnel de ce mail, je préfère ici vous donner un grand nombre d’informations bien détaillées, plutôt que de multiplier les mails plus personnels où je n’aurai pas pu aller autant en profondeur, j’espère que les quelques lignes qui suivent répondront aux questions que vous m’avez posé ces derniers mois.

\par
\par
Depuis que je suis revenu à Bali mon projet a été de participer à l’agrandissement des bungalows et restaurant tenus par un ami de longue date, Sales (de son vrai nom Nyoman Parna). Sales était pêcheur quand je l’ai connu, essayant de pêcher quelques maquereaux chaque jours, avec de gros risques de casse pour sont bateau vu les vagues et les cailloux en bord de mer. Il avait la vie que tout le monde a encore ici, c’est à dire qu’il vivait de presque rien, dans une maison faite de quelques moellons et de taules, sans eau courante, toilettes, douche, ni électricité.

\par
\par
Cette vie est toujours celle de la majorité des personnes dans le village de Seraya. Ne s’en sortent grâce au tourisme ceux qui parlent anglais ou qui peuvent conduire une voiture. Sinon s’en sortent ceux qui arrivent à décrocher un job de fonctionnaire (par de la corruption, pour décrocher le poste).

\par
\par
Seraya : Signifie « les rebelles ». Bali est Hindoue, il y a donc comme en Inde un système de castes. En règle générale lorsqu’un balinais rencontre un autre balinais qu’il ne connait pas il s’adresse à lui dans une langue très respectueuse (disons que c’est un vouvoiement++, avec des mots particuliers) ils s’accorderont ensuite sur le langage adapté à la caste de la personne en face dès qu’ils la connaitront. Historiquement, à Seraya les gens n’aiment pas s’encombrer des ces manières, et ont très tôt décidé de ne pas en tenir compte et parler directement dans le langage des Sudra (la caste la plus basse, 90\% de la population). Petit à petit, les trois castes dirigeantes (Brahmana, Satria et Wesia) ont su qu’elles ne seraient pas traitées avec les honneurs de leurs rangs dans ce village et sont parties. J’ai de mon coté plusieurs explications pour expliquer cet esprit rebelle, mais ce qui compte surtout c’est de savoir que du coup, les gens ont des fortes personnalités dans ce village, petite introduction à la suite !

\par
\par
J’ai toujours gardé contact avec Sales depuis 7 ans, sans pour autant l’aider plus que ca financièrement, mis à par quelques petites choses (qui sont déjà grosses à l’échelle balinaise), c’était plus un lien d’amitié, avec une famille qui m’avait touché.

\par
\par
Il y a deux ans je suis revenu le voir et j’ai eu l’agréable surprise de découvrir qu’il avait construit un bungalow pour accueillir des touristes, gages d’un changement de vie et de meilleurs revenus pour lui. Il avait aussi commencé les fondations pour construire un restaurant. J’en ai profité pour dormir dans le bungalow avec Florian, le pote qui voyageait avec moi, j’étais très content de voir ce changement de vie. Il avait par ailleurs ouvert un warung pour sa femme. Les warungs sont des petites boutiques vendant de tout et de rien aux personnes du village (savon, riz, pétrole, …). ça lui permettait de vivre mieux qu’avant.

\par
\par
Puis cette année, beaucoup de choses ont changé pour moi, autant au niveau professionnel qu’au niveau personnel, j’ai eu besoin de faire un gros break, rien n’allait plus. J’ai donc fait une rupture conventionnelle avec mon boulot et je suis parti prendre l’air, la destination était toute indiquée… Bali.

\par
\par
En arrivant j’ai pu voir que la vie de Sales avait encore bien changée, il y avait maintenant 2 bungalows, et le restaurant était fonctionnel.

\par
\par
Il m’a indiqué alors que l’argent pour construire tout ça ne venait pas de lui, mais de Jörg, un ami allemand qu’il connaît depuis 10 ans et qui vient à Bali 2 fois par ans. C’est cet ami qui a souhaité investir pour aider le plus de personnes dans le village, et en premier lieu Sales. Sans Jörg, Sales serait encore pêcheur.

\par
\par
Les bungalows et le restaurant sont regroupés sous le nom Frangipani Inn (ce nom est tiré du nom du bateau par lequel on s’est rencontrés, Sales et moi il y a 6 ans). <- Ca c’est pour l’anecdote !

\par
\par
En voyant que l’affaire marchait bien, et en voyant mon accueil dans le village j’ai voulu faire parti de l’aventure. Sales en était très content, et Jörg qui était à Bali à ce moment l’était encore plus, enchanté d’avoir un soutient occidental dans son projet qu’il tenait à bout de bras au niveau organisationnel et financier. On a alors convenu que je rentrait en France pour régler différentes choses (comme rendre mon appart, …) et retour à Bali en juillet. L’idée était alors d’aider Sales à mettre plus de règles dans la gestion, pour préparer le terrain à mon arrivée dans l’affaire et à l’investissement d’un troisième bungalow. Toutes les règles que j’ai souhaité mettre sont les règles que Jörg avait souhaité mettre depuis longtemps, mais qu’il ne pouvait pas imposer depuis l’Allemagne, il s’était donc fait une raison la dessus. Il s’agissait simplement dans un premier temps de noter toutes les dépenses et toutes les recettes, pour avoir un vrai suivi financier et avoir une idée plus claire de l’argent généré par l’affaire. Jusqu’ici, le porte-monnaie familiale était confondu avec celui du warung et celui des bungalows et restaurant. Mais Jörg et moi ne faisions parti que des bungalows et restaurant, il fallait donc séparer la gestion de l’argent rapidement, première étape difficile à faire accepter.

\par
\par
A cela, rapidement après mon arrivée, un très bon cuisinier, Kadek, qui travaillait dans des villas de luxe juste en dessous a démissionné. Il était hors de question de laisser filer cette occasion pour le restaurant. Kadek est le cousin de Sales, mais c’est surtout un mec qui a de l’or dans les mains, il est un excellant cuisiner, mais a aussi été très bon comme sculpteur sur bois dans le passé, et sait très bien bricoler. Pour Frangipani Inn, c’était une occasion à ne pas rater !!! Mais du coup cela a accéléré la nécessité de tenir des comptes justes, un gros salaire apparaissait dans la liste des dépenses, rien à voir avec le salaire de la personne qui nettoie les bungalows !

\par
\par
Les règles mises en places pour aout n’ont pas été respectée du tout, bien sur toutes les dépenses ont bien été reportées mais un grand nombre de recettes ne sont jamais arrivée dans la boite, restant bloquées dans la poche de Sales. Début septembre, nous avons eu une grosse réunion avec Sales et Kadek (grosse veux dire supérieur à 1/2h, la capacité de concentration de Sales étant complètement terminée après ce délai, il se contente de dire oui et fera non plus tard). Cette réunion a duré 3h et la conclusion a été qu’on oubliait le mois d’aout (qui a été full booké, et pour lequel les comptes ont terminé en négatif… sans commentaire), mais que plus aucune erreur ne serait tolérée en septembre, sous peine que je coupe tout me concernant. Pendant ce temps, le nouveau bungalow commençait, les pierres pour fondations étaient déjà cassées, le sable livré, le terrassement fait. J’ai donc dit stop pour le bungalow, en attente de voire le résultat fin septembre.

\par
\par
A la fin septembre Jörg est revenu d’Allemagne, et on a eu plusieurs grosses discussions a propos de l’avenir. Les règles de comptabilité n’ont pas toutes été respectés pendant le mois, mais on s’accordait avec Jörg pour dire qu’avec un contrôle plus strict les derniers manquements pouvaient être jugulés. Cela passait par le fait d’écrire tous les tarifs publiquement dans les bungalow pour que cela arrête de fluctuer selon la tête du client (et la dette du patron), les commissions étant connus pour ces tarifs, on savait ce que devait donc rapporter chaque tour ou activité vendus a un client, et Sales ne pouvait plus nous arnaquer la dessus.

\par
\par
Les commissions sont une pratique généralisée à Bali. Si vous venez avec un guide dans un restaurant ou un hôtel, le guide recevra au minimum 10\% de ce que vous dépensez. Cela peut monter à 40\% dans les zone à grosse concentration touristique où la concurrence faire rage, voire même à 60\% pour les bijoux. Le fonctionnement est quasi toujours le même, le client paie à la boutique, au restaurant ou a l’hôtel, et la commission passe dans le dos pour tomber dans la poche du guide. Pour la vente d’un ticket de bateau, Sales appelle la compagnie et dit le prix qu’il a vendu le ticket. Le client paie ce prix à la compagnie le jour J croyant qu’il s’agit du vrai tarif, et que Sales a juste rendu service. Plus tard, Sales se pointe au guichet et reçoit la différence entre le prix normal et le prix vendu. Pour un ticket à 250.000 Rupiahs (tarif normal pour Amed -> Gilies en fast boat), Sales n’avait pas de scrupules à vendre jusqu’à 600.000 Rupiahs, se faisant donc plus de 50\% de marge pour un simple coup de fil. Ces montants ne vous disent rien, mais sachez que pour un couple réservant un aller retour, la marge de 1.400.000 Rupiahs faite en moins de 2 minutes avec investissement 0 et risque 0 correspond à un mois de salaire d’un banquier.

\par
\par
Il est normal de se faire des marges dans le tourisme, il y a des factures à payer, mais il n’est pas normal de se faire plus de 50\% sur le dos des touristes à qui on fait de grands sourires et pour qui on passe pour un ami. Pour l’exemple ci dessus j’avais fixé le tarif à 400.000 Rupiahs (donc 150.000 Rupiahs de commission). Un tarif en dessous étant difficilement acceptable par la compagnie, car ça casse la concurrence quand les autres touristes voient un client payer si peu cher.

\par
\par
Voila, en gros il a fallu mettre un grand nombre de choses au point qui ont toutes été difficiles à accepter (que ce soit de la marque de l’eau qu’on vend, du PQ, jusqu’aux commissions, relations avec les chauffeurs qui font les tours, et pleins d’autres choses. Sales voulait toujours payer le prix le plus bas quitte à acheter de la merde, mais le facturer au tarif le plus haut possible. Il en est venu à faire payer le même prix pour un kit masque palme tuba que pour un scooter. Le scooter étant déjà au maximum acceptable de 60.000 Rupiahs/jours, je refusais qu’un bout de plastique soit loué au même prix).

\par
\par
A la fin des discussions, qui ont duré 5 heures fin septembre (soit 10 fois la capacité de concentration de l’intéressé), la conclusion précisait la répartition financière dans le future (40\% Sales, 40\% Jörg et 20\% moi, fonction du travail fourni et du coût de l’investissement de chacun. Cette répartition était plus qu’équitable pour Sales, car oubliant les coûts de gestion du booking internet pour Jörg et moi, et autorisant des entrées d’argent directement liées à Frangipani mais allant dans l’unique poche de Sales (laundry, location de scooter et équipements de plongée), concessions faites pour apaiser la situation. La deuxième conclusion était que toute grosse évolution ou dépense d’argent devait avoir l’aval des 3 personnes. Jusque là, rien d’illogique…

\par
\par
Oui, mais…

\par
\par
2 jours plus tard Sales est allé dans sa famille à Singaraja (3h d’ici) pour une cérémonie religieuse. Normalement il y va en scooter, facile et moins cher que la voiture (dont il ne disposait pas). Seulement cette fois il a cherché un moyen alambiqué de se faire emmener en voiture par quelqu’un qui devait faire un gros détour pour ça. Le doute s’installe. Le lendemain il nous appel en disant qu’il reviens avec un voiture de son oncle, pour laquelle il n’a que les taxes annuelles à payer et avec laquelle il voudrait conduire les client pour les tours.

\par
\par
Dès son retour, Jörg et moi avons fait une grosse levée de bouclier pour les raisons suivante :

\par
\par
1. Ce changement structurant n’a pas été discuté avec nous, tout au plus il m’avait parlé d’un autre plan voiture 2 mois avant, et j’avais dit non car il ne prenait pas les couts en compte, seulement l’appât de l’argent rapporté par les tours, sans conscience du coût d’entretient.

\par
\par
2. Sales est un exécrable conducteur. Il est la seule personne que je connaisse qui ai réussi à me faire peur au volant. Dans le contexte balinais ou le permis de conduire s’achète mais ne se passe pas, et la dangerosité de la route qui en découle, il était hors de question de lui confier des clients.

\par
\par
3. Frangipani fonctionnait déjà avec deux conducteur de confiance, très compétents et souples, qui ont eux aussi besoin de l’argent généré par les tours. Il était hors de question de leur faire ce coup la.

\par
\par
4. Le rôle de Sales sur lequel nous étions d’accord était d’être manager de l’ensemble, ce rôle étant incompatible avec le fait de tourner autour de Bali dès qu’un client en a besoin. Sales devait être la personne qui s’assure au quotidien que tout marche bien, et qui accueille de potentiels nouveaux clients qui arriveraient par hasard, sa femme ne parlant pas anglais.

\par
\par
5. Sales a fait toute une série de mensonges adaptés à chaque personne concernant cette voiture. Sont idée était que si il n’était pas autorisé à conduire les client, qu’il emploi un chauffeur au salaire journalier très bas, utilisant cette voiture, et Sales se faisant la marge réelle générée par le tour (évidement pas Frangipani, la voiture n’étant pas cautionnée par Jörg et moi).

\par
\par
La levée de bouclier a été très courte, je pense qu’il avait déjà prévu sa réaction face à nous. Il nous a foutu dehors sans se rendre compte des conséquences.

\par
\par
Le terrain est la propriété 100\% de Sales, la confiance étant la règle, jusque là. Jörg n’a aucun droit de propriété sur ses investissement, mais tient Sales par le fait que 100\% des bookings sont générés par internet, l’expulsion de Jörg rend donc les bungalows vides, Sales étant incapable de toucher un ordinateur pour autre choses que regarder ses mails une fois par semaine (alors qu’il devrai le faire tous les jours). Les sites de booking (Airbnb et Wimdu) étant ce qu’ils sont, il faut du temps pour arriver en tête des résultats de recherche. Ces le cas actuellement, mais même si Sales se découvrait une passion subite pour l’informatique, il lui faudrait au minimum 2 ans de travail acharné pour arriver au niveau de booking actuel.

\par
\par
Après cette réaction, je suis allé boucler mes affaires qui étaient dans se maison, et suis revenu vers lui pour essayer de comprendre son comportement. Et la il m’a ouvertement et publiquement dégagé. Ses cris n’étaient plus destinés à moi mais aux 40 personnes qui étaient autour (sculpteurs sur bois, jeunes qui jouaient au volley, voisins, …). Il a voulu montrer publiquement qu’il était le maitre.

\par
\par
Oui, mais…

\par
\par
Ca n’a pas marché ! Le téléphone balinais est encore plus efficace que le téléphone arabe, et tout le monde nous adore Jörg et moi dans le village. Ca, et la conséquence de devoir gérer tout seul les bungalows, ce sont les deux choses qu’il avait sous estimé… et pas qu’un peu !!!

\par
\par
On s’est rapatrié le soir chez Kadek (qui est donc le cuisinier, le cousin de Sales, mais aussi un amis de longue date de Jörg et moi). Kadek ne cautionne pas du tout ce que sont cousin a fait, et est 100\% de notre côté.

\par
\par
Ce soir la a été sans doute le moment le plus émouvant de toute ma vie.

\par
\par
Chez Kadek, toutes les personnes de la communauté sont arrivées nous soutenir. C’est une société patriarcale, l’homme représente donc la famille. Et ce soir là c’est entre 40 et 50 personnes qui sont venue nous voire pour nous soutenir, principalement des hommes, certaines familles au complet. Personne ne comprenant ce qu’il se passait et pourquoi on avait été mis dehors.

\par
\par
De l’avis de plusieurs amis, ce qui s’est passé ce soir là est totalement inédit. C’est quelque chose qui se passe de manière plus mesurée, uniquement lorsqu’il y a un décès. L’atmosphère ce soir là était vraiment spéciale, et plusieurs personnes nous ont expliqué plus tard que ce qui s’était passé montrait un profond respect de toute la communauté pour nous, et qu’ils avaient eux même été étonné par l’importance que ça a pris.

\par
\par
Le lendemain, ayant eu la nuit pour réaliser les conséquences de sont acte (et comprendre que tout le monde était venu nous voir, donc personne de son côté), Sales est venu faire des excuses à Jörg (mais pas à moi !!!), lui demandant de revenir dans l’affaire Frangipani, disant que la voiture serait rendue à sont oncle, qu’il n’avait pas eu à faire d’emprunt pour l’avoir donc que tout pouvait se déboucler sans conséquence. Jörg a simplement posé les conditions qu’il y aurai à sont retour, SI il revenait, sans donner de réponse claire.

\par
\par
Les jours suivants Jörg et moi avons navigués entre deux maisons, celle de Kadek et celle de Komang, l’un des deux chauffeurs. Puis on est allés sur Nusa Lembongan pour prendre l’air et réfléchir à la suite à donner à ça (une île voisine de Bali).

\par
\par
Durant ces jours, on a appris que finalement il n’était pas possible de rendre la voiture et que Sales avait fait un emprunt gigantesque pour l’acheter (100 millions de Rupiahs). Pour faire une comparaison compréhensible, le prix des voitures n’étant pas baissé en fonction du niveau de vie indonésien, forcément c’est de l’import, donc ca coûte encore plus cher. C’est à peu près comme si vous faisiez un emprunt de 100.000\euro  pour acheter une vieille voiture que vous ne savez pas conduire, qui consomme beaucoup trop pour être louée au tarif du marché et, comme vous n’êtes pas mécano vous n’avez pas su déceler ses problèmes. Dans les 15 jours qui ont suivi, Sales a déjà eu plus de 2.000.000 Rupiahs de frais dessus. (Pour notre niveau de vie, ça représenterait 2.000\euro ).

\par
\par
Précision sur l’emprunt : à Bali les banques prêtent à 2\% mensuel (!!!).

\par
\par
Petit exercice de math : Je veux acheter un scooter d’occasion qui coûte 12.000.000 Rp, pour cela j’emprunte à 2\% mensuels la totalité de la somme. Ma capacité de remboursement est de 250.000 Rp mensuels (montant déjà important), au bout de combien de temps le scooter sera t-il à moi ? * Réponse : un peu plus de 13 ans. Allez, on refait le même exercice avec la voiture de Sales et en reversant tous les hypothétiques salaires qu’il puisse avoir sans rien garder pour manger : endettement à vie.

\par
\par
Jörg ayant déjà fait trop d’investissement sur le terrain de Sales et tous les voisins (sculpteurs, chauffeurs, pêcheurs, …) comptant sur un apport de tourisme, il décide de finalement revenir et donner une dernières chance à Sales, avec une forte tutelle de Kadek (cuisinier donc) et Komang (chauffeur, très rigoureux). Au moindre écart dans les finances, tout lâcherai. Jörg se retirerai de l’affaire, Kadek ne voudrait plus travailler avec son cousin tricheur, et Komang se retirerai aussi.

\par
\par
De mon coté, il est simplement hors de question d’imaginer un instant revenir dans cette affaire. Je suis une personne très souple, je peux tout accepter et pardonner, laisser plusieurs chances à une personne de s’améliorer avant de perdre ma confiance. Mais une fois que la confiance a disparu, c’est définitif !

\par
\par
Ensuite, je suis revenu dormir quelques nuits dans un des bungalows à la demande de Sales. Je ne l’ai pas fait par amitié, mais pleinement conscient que ça l’aiderai à sauver la face vis à vis des voisin. Si jamais Jörg partais de cette affaire ensuite, les voisins se retourneraient contre Sales, c’est la communauté balinaise. Plus personne ne viendrait acheter au warung de sa femme, les cérémonies alentours seraient de plus en plus bruyantes, surtout en période de booking, avec des enceintes puissantes dirigées vers les bungalows. Sales perdrait ses client et tous ses revenus. Plus personne ne viendrait aux cérémonies de sa famille et il ne serait plus le bienvenue dans les autres cérémonies. Ces réactions en chaines possible m’ont été racontées par des personnes qui sont très mesurées en général, j’ai une très forte tendance à les croire. Alors si pour moi l’épisode se termine finalement sans trop de casse (entre 1.000 et 2.000\euro  de perdus, c’est les 10 prochaines années de sa famille qui se jouent en ce moment. Donc je ne mettrait plus un sous dans l’histoire, mais je ne lui souhaite quand même pas les pires conséquences, j’acceptai ces jours de ravaler ma langue.

\par
\par
… ce qui me vaut le surnom de « Ba Boedoeh », celui qui est fou. (Ca y est maintenant vous savez d’ou ça vient !!!)

\par
\par
Kadek, qui m’héberge depuis ces évènements et qui prévoit d’ouvrir une boulangerie chez lui en parallèle pour les locaux et les touristes (je l’ai aidé pour la construction, toit, carrelage, mur d’enceinte, …), a même prévu de l’appeler « Ba Boedoeh Bakery », la boulangerie du fou :) Et ca me touche;

\par
\par
J’ai mangé la première bouché de pain de la boulangerie hier matin… mmmm… un petit déjeuner avec du pain… ça faisait 4 mois que ca m’était pas arrivé !!!

\par
\par
Jorg est parti il y a 2 semaines maintenant. Le principal deal qu’il avait passé avec Sales pour la dernière chance était que Sales ne conduise pas les clients. Dès le lendemain Sales a pris le volant, ainsi que les deux jours d’après pour conduire les clients autour de Bali, il n’a même pas attendu 24h pour casser le deal, croyant naïvement que l’information n’arriverai pas jusqu’en Allemagne.

\par
\par
Le lendemain je suis retourné voir Sales pour lui demander quel était son plan pour me rembourser une somme que je lui avait prêté avant le clash, j’avais fait la banque pour lui éviter de payer les 2\% mensuel d’un prêt bancaire ici. Sales me devait donc 10 millions de Rupiahs, et m’a clairement dit qu’il ne m’en rembourserait pas 1 seule Rupiah. Sur ce j’ai donc informé Jörg de l’évolution de la situation concernant le prêt et la conduite des clients, lui demandant d’appeler Kadek et Komang pour confirmation. Dans la demi heure les sites de booking étaient mis en stand by, les pages n’étaient plus accessibles et les bookings futurs étaient annulés et remboursés.

\par
\par
Aujourd’hui, je suis donc dans une situation étonnante ici. Mes parents sont à Bali jusqu’à demain, leur visite était prévue de longue date. Ensuite je vais rejoindre des amis qui étaient aussi venus me voir et qui avaient prévu leur visite de longue date eux aussi. On va faire de la plongée ensembles.

\par
\par
En parallèle, je pense toujours très sérieusement à monter quelque chose ici. Je connais maintenant toutes les personnes qui seraient à la fois indispensables, disponibles et motivées pour travailler ensembles avec moi. Toutes ces personnes (Kadek, Komang, Rina, Ketut, et d’autres …) sont toutes des personnes de valeur, qui travaillent intelligemment, sont complémentaires et méritent qu’on leur fournisse un bon environnement de travail pour donner la pleine puissance de ce qu’ils savent faire. Les deux principaux étant Kadek (excellent cuisinier et bricoleur) et Komang (Très rigoureux, ancien (et excellent) cuisinier dans un paquebot de luxe 5 étoiles autour du monde, et chauffeur ayant sa propre voiture). Ces deux personnes sont, pour ne rien gâcher, de très bons amis depuis plus de 17 ans, chacun fait parti de la famille de l’autre.

\par
\par
C’est l’Indonésie, il n’y a pas d’argent mais un climat de rêve, des gens supers, un art des belles choses et un afflux de touristes comme peu d’endroits dans le monde.

\par
\par
Seraya est un village de côte traditionnel, avec pêcheurs, sculpteurs, et très peu de tourisme encore. Apporter ici la dernière pierre qui permettrait de faire marcher toute une équipe ensemble, dans une bonne ambiance, et faire vivre autant de familles tout en faisant connaître l’artisanat de pleins d’autres familles du village en sculpture et osier, les aidant à vendre pour un prix cohérent leur travail au lieu de se faire avoir par une chaine d’intermédiaires serait un super projet.

\par
\par
Ce projet coute de l’argent, je ne peux pas le faire seul. Je voudrais en ce moment le monter avec l’aide de Jörg qui serait peut être prêt à repartir dans une nouvelle aventure.

\par
\par
Pour vous donner une idée du prix d’un tel projet, le montant nécessaire pour lancer quelque chose serait d’environ 50.000 à 60.000\euro .

\begin{itemize}
     \item Le terrain couterait 20.000\euro  pour une dizaines d’ares (1 are = 100 m2)
     \item Il faudrait compter 20.000\euro  pour 3 bungalows
     \item Un restaurant avec une bonne cuisine vu le profil des lascars (je vous promet que ca envoi du gros quand on s’assoit à table !), environ 5.000\euro  à 10.000\euro
     \item Il faudrait compter ensuite environ 5.000\euro  pour être en règle avec les différentes licences touristiques.
\end{itemize}

\par
\par
Ces licences sont facultatives quand l’affaire appartient à un balinais, une bière pour le contrôleur suffit en cas de contrôle, mais dès que le propriétaire est étranger il devient très important d’être en règle car les problème grandissent rapidement (et de toute façon, les taxes Indonésiennes étant très basses il n’est pas question d’essayer de les contourner).

\par
\par
Kadek et Komang m’ont déjà tous les deux dit chacun de leur coté qu’ils seraient avec moi si je décidait de faire quelque chose. Il ont par contre chacun posé la même condition : que je monte ce projet seul ou avec Jörg, mais que personne de local ne soit propriétaire du projet, ils connaissent et aiment la méthode de travailler ‘western’ mais ne veulent pas avoir un chef balinais qui essaierait d’imposer au jour le jour ses propres règles ou plutôt non règles et dérives. En claire, ils veulent qu’on fasse quelque chose ensembles, mais ne veulent pas retomber dans le cas Sales.

\par
\par
\dots Ca tombe bien\dots moi non plus !!!

\par
\par
Ces jours ci sont donc à la réflexion, j’ai déjà visité un terrain qui permettrait de faire un truc de malade avec une vue sur mer de fou (coté lever de soleil, avec les bateaux de pêche à voile qui rentrent le matin) ! Ce terrain est à un autre ami qui souhaiterait aussi voir un projet de Jörg et moi dessus, on serait donc prioritaire si on voulait le prendre.

\par
\par
D’un autre coté vu tous les évènements ces derniers temps, je pense qu’il est urgent d’attendre, Kadek a une boulangerie à lancer, Jorg et moi avec une réflexion à faire ensembles…

\par
\par
Voilà, j’espère vous avoir donné les nouvelles qui manquaient jusque là. Je suis vraiment désolé de ne pas avoir été régulier pour vous informer de l’évolution de la situation, mais je pense que vous comprenez que vu la complexité de la situation, les conditions matérielles ici au niveau d’internet, et les changements permanants j’ai vite arrêté d’informer autour de moi, étant mois même assez vite dépassé et n’ayant pas forcément toujours la volonté de réexpliquer, les bonnes nouvelles sont toujours plus faciles à donner ! Maintenant la situation semble plus claire, il y a de très bonnes bases pour monter un projet qui serait sain, avec de très bonnes personnes. Si cela se fait, ce sera sans doute rapidement, sinon, je rentrerais en France, mais vous avez bien compris que Bali restera longtemps dans ma tête, j’y reviendrai, et je suis certain que l’histoire n’est pas finie avec Kadek et Komang, même si ça doit attendre.

\par
\par
Et vous, vous en pensez quoi ?

\par
\par
Etienne

\par
\par
// TODO : Ajouter des photos dans tous l'article, des schémas, des idées...

\par
\par
%\end{multicols}

\vfill
