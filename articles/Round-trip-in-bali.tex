layout: landscape
title: Round trip in Bali
date: 2008/04/14 18:20:01
tags:
---

Oulala... que de choses a dire...

Je vous ai laissés à la fin de ma première semaine à Bali. Ma deuxième semaine à été très mouvementée, car j'ai loué un scooter pour faire un tour de l'ile, en 4 jours.

Me voila donc parti sur les routes, avec pour seule carte un plan de l'ile qui tiendrais presque sur une seule main, sur laquelle figure les grandes routes, pour les reste, la discussion avec les gens m'a bien suffit à m'orienter.

Alors premiere precision : parlons de la circulation ici... Nous vous avions parlé avec Cécile de la circulation en Inde, je n'aurais jamais loué une scooter en Inde, trop dangereux pour un novice. Ici c'est différent... quoi que... Les intersections sont respectées, les feux aussi, mais par contre il faut savoir qu'il y a réellement deux types de circulations qui cohabitent ici : les voitures qui font ce qu'elle peuvent pour respecter le code de la route, ce qui ne les empêchent pas de s'arrêter en plein virage et au milieu de la route sur ce que l'on appelerait une route départementale. Et les 2 roues, qui font ce qu'ils veulent : On peut doubler par la gauche, par la droite, slalomer, rouler à contre sens... don't worry, be happy. De temps en temps ca a du bon d'être en deux roues...

Mais que fait la police !!! Réponse : elle encaisse les backchiches. On a beaucoup plus de chances de se faire arrêter ici quand on est "western" (blanc) que quand on est Indonésien. Pourquoi ? Parce qu'on est plus solvable bien sûr... La majorité du temps le seul et unique but du policier qui vous arrête est que vous lui donniez un billet, on peut rouler sans permis ici, du moment qu'on a 50.000 Rp en poche. Quand je parle de contrôle je ne dis pas ça en l'air, je me suis fait contrôlé trois fois ce soir même sur une zone de moins de 50m de diamètre, sans éxagérer, mais je n'ai pas eu a sortir de billets, j'étais (presque) en règle.

Nous fermons la parenthèse code la route pour cette fois-ci.



Le premier jour je suis allé dans un coin de surfers, en dessous de Tabanan, cette journée a surtout été pour moi la découverte avec la conduite indonésienne, à gauche, et son code de la route folklorique. Le soir je me suis posé à une terrasse, à discuter, je suis tombé sur des Français. Enfin quand je dis Francais, c'est d'origine, car ils voyagent tellement qu'ils n'ont jamais vu la couleur d'un euro ! La vie est belle quand on se contente de trois sous pour vivre, et ainsi voyager partout dans le monde.

Le lendemain, remontée sur Singaraja à travers de magnifiques rizières, c'est un coup à avoir un accident tellement je tournais la tete pour voir les superbes paysages qui défilaient.









Quand j'ai pris cette photo, il y avait un bruit au loin, j'ai mis du temps à comprendre qu'il s'agissait d'un homme, dans la cabane du bas, qui me faisait signe, et qui faisait du bruit pour que je le remarque. Un signe de ma part et il s'en est retourné dans sa cabane, il a du penser qu'il serait sur la photos. Les gens ici sont aussi tellement gentils que l'on a vraiment envie de leur faire plaisir.

Singaraja : pour sur ce n'est pas la meilleure étape, coin touristique si il en est, j'ai quand même réussi à me trouver un petit endroit à l'écart pour dormir, mais c'était assez difficile car, ne voulant pas paraître comme ignorant quand on leur pose une question, les gens ici vous repondront toujours une direction et une distance lorsque vous cherchez un "cheap hôtel", le tout étant apres de savoir si ce que l'on vous a dit est vrai ou pas. Il m'est arrivé que l'on me réponde alors que la personne ne parlait pas anglais, et n'avait donc pas compris la question.

Apres Singaraja, direction Seraya, au bout de la pointe de Karangasem. Je suis allé dormir chez Sales, amis de Patrick, qui m'a accueilli comme un roi, nous avons passé une super soirée avec sa famille et ses amis à boire un coup.

Sales et l'un de ses deux fils, qui a adoré les séances de guilis prodigués par mes soins.  



Et vous savez quoi... ? j'ai eu un super petit dej le lendemain matin, une noix de coco qui était encore à 15m de haut 10 minutes avant que je boive son jus... si c'est pas la classe...



Je suis redescendu de Seraya avec Sales en scooter, puis, en repartant demain, nous le déposerons près de chez lui après environs 7 heures de voile.

C'est maintenant les opérations de remplissage des soutes, avec de l'eau, du carburant, et de la nourritures. Puis dire au revoir, surtout pour Patrick qui a été ici durant 7 mois. De mon coté aussi c'est un peu dur de partir, il y a déjà des personnes que j'aimerais revoir.

Direction Madura, sur l'île de Java pour les jours à venir, puis Bornéo, plus au nord. Et de Bornéo nous mettrons le cap sur Singapour, où je dois arriver avant la fin de mon visa Indonésien (fin du mois d'avril).

Voila, j'éspère que suivre mon voyage vous plaît toujours, je vous dis à très bientôt.
