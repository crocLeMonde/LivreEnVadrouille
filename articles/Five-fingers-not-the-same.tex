\section{Five fingers, not the same}

Date: 03/02/2008

\begin{multicols}{2}

Et oui, Incredible India !

Ce matin, on était parti pour visiter le Taj Mahal, enfin plutôt Mathura en fait, une ville plus au nord, quoique d'un autre côté pas énormement de soleil, donc la balade en rickshaw autour de Agra semblait une bonne idee. Mais alors quand je suis descendue dans la rue ce matin, je suis carrément tombée amoureuse! Agra, chaleureuse, plus calme et colorée que Delhi, très aérée et vivante... c'est décidé, ce sera une balade dans le quartier et puis oh, visite du fameux Taj Mahal (situé dans cette même ville) car le soleil est revenu.

Mais... Incredible India! Petit déjeuner fameux avec vue sur "une larme sur le visage de l éternité" et rencontre avec des jeunes. Ils jouent au cricket, ils sont accueillants et curieux. Merci a Vipin pour la balade sur les toits, la vue imprenable sur le Taj, ses portes et ses fortifications. Cependant attention à ne pas se montrer, les policiers ne permettent pas que les touristes approchent par les toits! Sécurité pour les Indiens nous dit-il. Ils nous protègent, on les écoute, on les respecte. Puis le café indien, en tailleur par terre, avec toujours la vision de rêve, une oeil sur les toits et les rues, le muezzin en fond sonore et le soleil qui caresse la peau.

Quel temps fera t-il le lendemain? Vipin ne le sait pas, son pays est une terre de surprises, Incredible India!

Et les animaux, surprenants eux aussi. Les chiens et les vaches dans la rue, mais aussi les singes sur les terrasses, les écureuils qui pépillent, les rapaces qui paradent. Tout nous émerveille. On discute de beaucoup de choses, on échange sur nos cultures, nos pays. Les photos sont très appreciées, les gens fiers d'être pris et de se voir. Même dans l allée du Taj Mahal, une femme indienne me confie son enfant pour qu'Etienne immortalise le moment. Elle s'approche sans me connaître, me fait confiance, Incredible India!

Et même les femmes en sari, colorées à souhait, posent pour nous, les gens sourient, sont fiers, sûrement heureux d'être là. Nous aussi, on savoure, on se delecte, on profite (et on photographie, pour nous et vous mes amis). Le Taj, imposant monument, les pieds nus sur la pierre puis le marbre, le blanc et le rose, le bâtiment et le fleuve. Mais surtout, surtout, la beauté, le respect, la plénitude. Incredible India!

Demain est un autre jour, et il s'annonce tout aussi beau...

Vous n'avez pas le son, l'ambiance, les gens... Alors profitez bien des images !

To Vipin : We will always keep in mind this point of view : "Five fingers, not the same". Daniwad.

\end{multicols}
