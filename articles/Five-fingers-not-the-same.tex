\section{Five fingers, not the same}

3 fév. 2008

\begin{multicols}{2}

Et oui, Incredible India !

Ce matin, on était parti pour visiter le Taj Mahal, enfin plutôt Mathura, une ville plus au nord. Mais finalement vu le peu de soleil on a préféré aller se ballader dans Agra et ses alentours, on a décidé de remettre la visite du Taj Mahal pour quand il y aura plus de soleil. Mais alors quand je suis descendue dans la rue ce matin, je suis carrément tombée amoureuse! Agra, chaleureuse, plus calme et colorée que Delhi, très aérée et vivante... c'est décidé, ce sera une balade dans le quartier et puis oh, visite du fameux Taj Mahal (situé dans cette même ville) car le soleil est revenu.

Mais... Incredible India! Petit déjeuner fameux avec vue sur "une larme sur le visage de l éternité" et rencontre avec des jeunes. Ils jouent au cricket, ils sont accueillants et curieux. Merci a Vipin pour la balade sur les toits, la vue imprenable sur le Taj, ses portes et ses fortifications. Cependant attention à ne pas se montrer, les policiers ne permettent pas que les touristes approchent par les toits! Sécurité pour les Indiens nous dit-il. Ils nous protègent, on les écoute, on les respecte. Puis le café indien, en tailleur par terre, avec toujours la vision de rêve, une oeil sur les toits et les rues, le muezzin en fond sonore et le soleil qui caresse la peau.

Quel temps fera t-il le lendemain? Vipin ne le sait pas, son pays est une terre de surprises, Incredible India!

Et les animaux, surprenants eux aussi. Les chiens et les vaches dans la rue, mais aussi les singes sur les terrasses, les écureuils qui pépillent, les rapaces qui paradent. Tout nous émerveille. On discute de beaucoup de choses, on échange sur nos cultures, nos pays. Les photos sont très appreciées, les gens fiers d'être pris et de se voir. Même dans l allée du Taj Mahal, une femme indienne me confie son enfant pour qu'Etienne immortalise le moment. Elle s'approche sans me connaître, me fait confiance, Incredible India!

Et même les femmes en sari, colorées à souhait, posent pour nous, les gens sourient, sont fiers, sûrement heureux d'être là. Nous aussi, on savoure, on se delecte, on profite (et on photographie, pour nous et vous mes amis). Le Taj, imposant monument, les pieds nus sur la pierre puis le marbre, le blanc et le rose, le bâtiment et le fleuve. Mais surtout, surtout, la beauté, le respect, la plénitude. Incredible India!

Demain est un autre jour, et il s'annonce tout aussi beau...

Vous n'avez pas le son, l'ambiance, les gens... Alors profitez bien des images !

%To Vipin : We will always keep in mind this point of view : "Five fingers, not the same". Daniwad.

\end{multicols}

\bigskip
\textbf{\textsc{Commentaires}}

\medskip
Marie-Hélène a écrit le 3 fév. 2008 :
\begin{displayquote}
Super vos photos, vos commentaires, vous nous faîtes rêver! Nous vous accompagnons dans votre périple, et imaginons les bruits, les odeurs les couleurs et tout ce qui va avec la magie de ce pays!
Bonne continuation, continuez à faire de belles rencontres, et faites attention à vous.
Je vous embrasse !
\end{displayquote}

\medskip
Titou a écrit le 4 fév. 2008 :
\begin{displayquote}
Hey hey ! Ra ca donne envie d'y aller ! Arrêtez de nous narguer ça ne se fait pas ! Dire que dans 3 petites semaines c'est Belfort qui m'attend\dots Bon ok après 6 mois à l'autre bout du monde mais quand même ! Continuez de nous faire rêver et pour tout vous dire vous me donnez quelques idées de destinations pour mon ST50\dots
Eclatez vous comme des petits fous, faites vous plaisir et surtout faites gaffe à vous ! A plouche les amis !
\end{displayquote}

\medskip
Ewen a écrit le 4 fév. 2008 :
\begin{displayquote}
Salut Etienne ça va ça se passe	bien.
Bisous Moi et Nils pensons bien à toi.
\end{displayquote}

\medskip
Jean-yves a écrit le 4 fév. 2008 :
\begin{displayquote}
Nous suivons quasiment en temps réel cette plongée dans une société très différente. C'est fabuleux. Et puis en plus des images, il n'y peut être pas de son, mais il y du texte, du vrai, avec des petites pointes de poésie savoureuse. J'adore\dots
Il est vrai que La Savoureuse, ca vous connaît\dots
\end{displayquote}

\medskip
Gerien a écrit le 4 fév. 2008 :
\begin{displayquote}
Les gens ont vraiment l'air d'être super peace. C'est cool :)
Ça donne envie !!!!
\end{displayquote}

\medskip
Catherine a écrit le 6 fév. 2008 :
\begin{displayquote}
Merci pour tout ce que vous nous faites partager et à Cécile pour son mail.
J'adore la deuxième photo, nettement moins conventionnelle et plus parlante que celle que l'on voit habituellement. L'on aurait trop tendance à oublier que le Taj Mahal n'est pas le seul palais du lieu.
\end{displayquote}

\medskip
La vadrouilleuse a écrit le 10 fév. 2008 :
\begin{displayquote}
Il est vrai qu il y a d'autres curiosités à Agra que le Taj Mahal, dont le Fort Rouge qui m'a particulierement touchée. Mais pour info, on ne voit que le Taj sur la photo : il s'agit en effet autour des remparts, devant la porte sud, et au loin de deux annexes, le tout dans l'enceinte rose du Taj Mahal.
\end{displayquote}

\medskip
Tatid a écrit le 21 fév. 2008 :
\begin{displayquote}
Arf arf, plus de photos dispo, mais il me semble les avoir vues quand j'étais passé faire un tour sur votre blog =)
C'est fou qu'une mère confie son enfant comme ça à une inconnue, on verrait pas trop ça en France :-D ! C'est sympa que le temps soit toujours "au beau fixe" ! Profitez, ici à Paris, il pleut\dots ! :(
\end{displayquote}

\vfill

