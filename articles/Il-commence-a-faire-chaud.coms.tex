\bigskip
\textbf{\textsc{Commentaires}}

\medskip
Titou a écrit le 18 fév. 2008 :
\begin{displayquote}
Ra vous nous faites rêver même sans les images. Je comprend ce que vous ressentez car je l'ai ressenti ici aussi même si l'univers est complètement différent. C'est toujours très dur de transmettre ce que l'on ressent et que l'on voit parfois tellement c'est beau. Ce qui prouve que notre monde regorge de choses magnifiques dont on ne soupçonne parfois pas l'existence\dots Profitez en toujours à fond, faites de ce sejour un souvenir mémorable et on se revoit dans peu de temps. Faites gaffe à vous et à plus. Biz.
\end{displayquote}

\medskip
Lydie a écrit le 19 fév. 2008 :
\begin{displayquote}
Ahlala, vos paroles sont encore plus belles que les images. On se plonge avec "extase" à travers vos écrits !!!
Profitez-bien de vos derniers jours !!! Et j'attends de tes nouvelles ma poulette. Bisous.
\end{displayquote}

\medskip
Zan a écrit le 19 fév. 2008 :
\begin{displayquote}
Hébé, ça dépayse de la Franche-Comté\dots
Vous devez vraiment en avoir plein les yeux et les oreilles tous les jours (et ptet plein le c*l aussi des singes :D)!
Continuez de nous faire rêver avec ces photos, elles sont vraiment magnifiques!
@ bientôt!
\end{displayquote}

\medskip
Peggy a écrit le 20 fév. 2008 :
\begin{displayquote}
Etienne qui essaie d'apprendre aux singes les règles de bonnes conduites\dots j'adore!
\end{displayquote}

\medskip
Poun's a écrit le 20 fév. 2008 :
\begin{displayquote}
Salut vous deux. Bon, si vous continuez, on va affréter un avion pour vous rejoindre\dots
La météo, toujours pas de nuages? On a l'impression qu'il ne fait pas toujours très chaud là-bas, Vous n'avez pas froid?
Continuez les photos, même si elles ne passent pas toutes, et surtout mettez-en plein votre tête pour nous raconter tout ça,
Félicitations au nouveau tonton, et surtout à la maman et au papa,
Ca tombe bien, je travaille à Besançon vendredi, on va fêter ça!
A bientôt,
\end{displayquote}

\medskip
Tatid a écrit le 21 fév. 2008 :
\begin{displayquote}
J'ai toujours pas le net à l'appart, mais vu que j'ai pas grand chose à faire en stage, j'en profite pour ratrapper ma lecture de blogs (j'ai lu l'intégral de "Fab aux USA" en 2 jours :-) !
Vos photos sont vraiment superbes, c'est dingue le changement de culture par rapport à la notre, ça doit être vraiment enrichissant un voyage comme ça, puisqu'on est plus du tout dans la culture occidentale quoi ! Les monuments sont aussi impressionnants de beauté ! En tous cas, vous n'avez pas l'air triste (c'est le moins qu'on puisse dire !) et vous avez l'air d'en avoir profité au maximum, c'est génial !
Dis Dud, tu t'es fait des potes singes ? T'en ramène un, nan ?
\end{displayquote}

\medskip
Etienne a écrit le 21 fév. 2008 :
\begin{displayquote}
Alors question météo ça commence à chauffer. On a déjà du avoir du 30 degrés je pense. Merci bien pour les félicitation, je transmettrai avec grand plaisir à ma soeurette car elle ne doit plus trop avoir le temps de venir lire ce blog à mon avis\dots
Bernard, a vendredi ;-)
Tatid, on m'a même proposé de m'en vendre un à Agra, le mec rigolais devant ma tête de touriste à regarder le singe sur le toit de sa shop "You want it ? i give you for 100 Rs".
Continuez à nous écrire comme ça, ça fait très plaisir de voir que notre voyage est suivi. C'est à chaque fois le suspense quand on va sur internet. A bientôt.
\end{displayquote}

\medskip
Soeurette\dots a écrit le 21 fév. 2008 :
\begin{displayquote}
Comment ça j'ai pas le temps de vous lire\dots !!! je suis là\dots bon d'accord, je suis pas venue depuis un bon bout de temps, mais je rattrape mon retard ce soir pendant que Lucie dort.
Ce n'est pas l'endroit, mais j'en profite pour dire à tout ceux  qui me lirons que l'acouchement a été un peu dur, mais tout va bien maintenant. Ca fait du bien d'être chez sois\dots et Lucie a l'aire de se plaire chez elle\dots
Maintenant que j'ai parlé de moi, j'espère que vous reviendrez de ce voyage pleins de bons souvenirs dans la tête\dots j'imagine qu'à peine rentrés, vous allez avoir envie de repartir visiter le monde ?
Allez, continuez bien, à mardi prochain.
Bisou
Cécile la soeurette\dots
\end{displayquote}

\medskip
Gerien a écrit le 22 fév. 2008 :
\begin{displayquote}
Alors la grande question de la pause café de ce matin à l'Alstom c'est :
Est ce que dud c'est fait bouffer la main par un singe (surtout le gros là à la fin) ?
Allez, bonne fin de voyage les gens ;o)
\end{displayquote}

\medskip
Etienne a écrit le 23 fév. 2008 :
\begin{displayquote}
Héhé\dots je vois que ça bosse dur dans les bureaux d'Alstom\dots
Les singes étaient trop contents qu'on leur donne à manger pour être agressifs, en fait ils doivent avoir l'habitude car il y a beaucoup de passage là où on était. Mais par contre on en a déjà vu flanquer la frousse a une touriste jusqu'à ce qu'elle lâche la pomme qu'elle était en train de manger, et ça c'est assez impressionnant à voir, on doit rapidement se dire qu'une pomme ne vaut pas une morsure de singe.
\end{displayquote}

\medskip
Poun's a écrit le 23 fév. 2008 :
\begin{displayquote}
Salut à vous. N'oubliez pas de prendre des pommes, si ça peut vous sauver une main\dots
Pensez à ma commande : une vache sacrée, un singe, 1 l'd'eau du Gange et si vous avez un peu de place dans vos bagages, un éléphant me serait très utile.
A bientôt pour vos nouvelles aventures, j'espère que vous ne mourrez pas de chaud.
\end{displayquote}

\medskip
Gerien a écrit le 24 fév. 2008 :
\begin{displayquote}
Moi, ce que je commanderais prendrais moins de place\dots
:)
\end{displayquote}

\vfill
