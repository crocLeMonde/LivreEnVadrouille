layout: landscape
title: Une boucle du coté vert
date: 2010/12/07 17:25:09
tags:
---

Salut tout le monde, on repart faire une petite boucle à Mada, vers le nord cette fois, pour vous raconter la fin du voyage..

Nous étions restés à Tananarive où j'avas atterri après un vol depuis Fort Dauphin. Tana n'étant pas une ville très sympa à visiter, le lendemain j'ai pris un taxi brousse pour aller voir le parc d'Andasibe qui renferme l'un des lémures les plus rares de la Grande Ile, j'ai nommé l'Indri. Après moultes déboires de transports dus à une trop grande méfiance de ma part, j'arrive au parc en début d'après midi, à l'heure de la sieste des lémures. Qu'à cela ne tienne, si ils sont sympa ils auront choisi de faire la sieste à coté du chemin.. Je vois donc ce fameux lémure, le plus grand, en plein repos puis il se met à communiquer avec ses potes dans un bruit impressionnant (vidéos à venir une fois en France).

<img title="Indri, Andasibe, Madagascar, déc. 2010" alt="" src="http://etienne.croclemonde.org/public/madagascar/DSCF0386.JPG" />

On continu la promenade dans le parc, et je vois alors le lémure bambou, qui mange devinez quoi.. Je reste un certain temps à l'observer puis à le filmer, bien marrant ce petit lémure !

<img title="Lémur bambou, Adasibe, Madagascar, déc. 2010" alt="" src="http://etienne.croclemonde.org/public/madagascar/DSCF0395.JPG" />

Une fois finit la visite du parc me voici de retour à Tana le soir, puis direction le nord à partir du lendemain. La première escale sera Maevatanana.. oh et puis non il semble qu'il n'y ai pas beaucoup d'activité, vamos a Mahajunga direct, c'est parti pour 13h de taxi brousse pour arriver dans cette ville au bord de mer.

Je me ballade le lendemain, mais la ville semble déserte, en fait on est dimanche et le dimanche à Madagascar certains endroits on l'allure de ville fantôme. Voici quelques pousse pousse devant les boutres des pêcheurs, sur le port

<img title="Pousse pousses et boutres, Mahajunga, Madagascar, déc. 2010" alt="" src="http://etienne.croclemonde.org/public/madagascar/DSCF0403.JPG" />

Puis un peu plus loin sur la jetée je vois une pirogue qui rentre.

<img title="Pecheur, Mahajunga, Madagascar, déc. 2010" alt="" src="http://etienne.croclemonde.org/public/madagascar/DSCF0398.JPG" />

OK, sympa cette ville mais je n'ai pas envie d'y rester plus que ça. J'appelle alors Anthony, un français de l'Ardèche, que j'avais rencontré en début de voyage à Antsirabe, on s'était convenus de se donner des nouvelles vers la fin de ma boucle car nous serions alors tous les deux dans le nord du pays, l'occasion de voyager quelques jours ensembles. Et la, le hasard en voyage n'existe pas, il est en route pour Mahajunga et arrive dans moins d'une heure, cool rasta, on va pouvoir tracer le chemin ensembles.

Nous partons alors pour Nosy Be, ZE place touristique de Madagascar. Nous y passons deux jours, louons des motos pour faire le tour de l'île qui a certes, quelques belles plages, mais qui a surtout beaucoup trop de vazaha, et pas du meilleur spécimen. Il s'agit la du vieux blanc de 50 ans venus en colon profiter du rhum, des plages, et de certaines autres spécificités locales (certains comprendrons). Voici une plage du nord de Nosy Be.

<img title="Plage, Nosy Be, Madagascar, déc. 2010" alt="" src="http://etienne.croclemonde.org/public/madagascar/DSCF0418.JPG" />

Puis nous mettons le cap vers Nosy Komba, ile toute voisine et beaucoup plus authentique. Voici une pirogue croisée durant la traversée.

<img title="Pirogue, Nosy Be, Madagascar, déc. 2010" alt="" src="http://etienne.croclemonde.org/public/madagascar/DSCF0433.JPG" />

L'occasion de vous montrer un phénomène à Madagascar : dans tout le pays, certaines femmes se mettent une poudre sur le visage pour se protéger du soleil. L'effet est alors certaines fois étonnant pour l'observateur. Dans le nord du pays, et notamment à Nosy Komba, nous avons rencontré des femmes qui font de ce masque un élément de beauté, l'effet est alors magnifique, j'adore vraiment les deux photos qui suivent..

<img title="Sourire, Nosy Komba, Madagascar, déc. 2010" alt="" src="http://etienne.croclemonde.org/public/madagascar/DSCF0441.JPG" />

<img title="Sourire, Nosy Komba, Madagascar, déc. 2010" alt="" src="http://etienne.croclemonde.org/public/madagascar/DSCF0442.JPG" />

Nosy Komba est aussi l'occasion d'aller faire de la plongée sous marine, histoire d'aller voir ce qui se passe la dessous. Je n'ai pas de photos pour vous le montrer mais la vie sous marine ici est très active, les coraux sont magnifiques, de toutes les couleurs et les poissons tropicaux très variés, j'ai fais deux plongées de 50 minutes chacune, de toute beauté.

Nous quittons alors les îles pour aller à la pointe Nord de Mada, à Diégo Suarez où se termine mon voyage ces jours-ci. Parmi les beaux paysages à aller voir ici se trouve la mer d'Émeraude dont on m'a tant parlé, Anthony et moi louons alors un scooter pour se rendre à Ramena, point de départ des bateaux qui partent à la journée pour visiter cette baie protégée par une barrière de corail.

L'occasion de vous présenter Anthony, à gauche, ainsi que notre cuistot et notre barreur/pêcheur, deux bonhommes bien marrant qui nous ont préparé devinez quoi.. du poisson grillé pêché 10 minute plus tôt.. mmmmm !!!

<img title="Jack Sparrow, Mer d'Emeraude, Madagascar, déc. 2010" alt="" src="http://etienne.croclemonde.org/public/madagascar/DSCF0456.JPG" />

Allez, pour la postérité, "j'y étais !"

<img title="Etienne, Mer d'Emeraude, Madagascar, déc. 2010" alt="" src="http://etienne.croclemonde.org/public/madagascar/DSCF0457.JPG" />

Et on termine ce voyage à Madagsacar par une petite sieste, l'occasion de faire bon usage du cadeau d'au revoir de la conv'tek'team du tonnerre, merci à vous..

<img title="Hamac, Mer d'Emeraude, Madagascar, déc. 2010" alt="" src="http://etienne.croclemonde.org/public/madagascar/DSCF0463.JPG" />

.. et à bientôt dans un autre coin du monde, des suggestions ?
