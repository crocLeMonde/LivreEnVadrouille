layout: landscape
title: Fiche de la Malaisie
date: 2008/05/17 11:08:39
tags:
---

### Drapeau



### Situation
#### Présentation générale

La Malaisie couvre une superficie de 330.000 km² et comprend deux ensembles géographiquement distincts séparés par 800 km de mer :  Une partie péninsulaire de 131.000 km² (750 km du nord au sud avec une largeur maximale de 350 km), montagneuse mais d'altitude peu élevée, partageant 506 km de frontière avec la Thaïlande au nord et bordée par Singapour au sud
Une partie insulaire, formée de deux Etats au nord de l'île de Bornéo (Sabah et Sarawak), enclavant le Sultanat de Brunei et limitée par une frontière de 1782 km avec l'Indonésie. Largement ouverte sur la mer, la Malaisie compte 4675 km de côtes. La partie péninsulaire est baignée à l'ouest par le détroit de Malacca qui la sépare de l'île indonésienne de Sumatra ; à l'est se trouve la mer de Chine méridionale. Le Gunong Tahan (2187 m) est le point culminant dans la péninsule ; au Sabah, le mont Kinabalu (4101 m) représente le premier sommet de l'Asie du Sud-Est. Les deux ensembles qui composent la Malaisie ont en commun un climat équatorial, constamment chaud et humide, et un couvert forestier dense.

Les 11 Etats de la Malaisie péninsulaire sont :  ouverts sur la côte occidentale : Perlis, Kedah, Perak (frontaliers avec la Thaïlande), Penang, Selangor, Negeri Sembilan, Melaka Ouverts sur la côte orientale : Kelantan (frontalier avec la Thaïlande également), Terengannu, Pahang au sud : Johor.

### Population

Des 24,5 millions d'habitants peuplant la Malaisie, environ 85% sont concentrés dans la partie péninsulaire où ils sont inégalement répartis (de 22 hab/km² dans Pahang, l'Etat présentant la plus grande superficie, jusqu'à 926 hab/km² dans Penang, au nord ouest). Selangor (sur le détroit de Malacca) est l'Etat le plus peuplé avec 4 millions d'habitants. Un tiers de la population totale est âgé de moins de 15 ans. La population active représente près de 10 millions de personnes. Le recensement 2000 dénombrait 65% de Bumiputras - "fils du sol" - 26% de Chinois et 7,7% d'Indiens. Un petit nombre de populations aborigènes subsiste en Malaisie (Orang Asli et tribus du Sarawak et du Sabah).

|                                       |     MALAISIE    |    FRANCE    |
| ------------------------------------- |: -------------: | -----------: |
|Population                             |    24,5 millions| 59,2 millions|
|Densité                                |     70,7 hab/km²|   108 hab/km²|
|Accroissement naturel de la population |               2 |          0,4 |
|Indice de fécondité                    |               3 |          1,8 |
|Espérance de vie                       |           72 ans|      78,5 ans|
|Urbanisation                           |             62 %|         75,6%|

### Climat

Le climat est de type tropical humide avec des températures quasiment uniformes tout au long de l'année (minimum : 25°, maximum : 37°). Les précipitations les plus importantes sont reçues entre septembre et février (moyenne annuelle : 2700 mm). Cependant, un mois "sec" comporte encore près de 100 mm de pluie. L'hygrométrie est en permanence à plus de 80 %.

### Villes principales
#### Kuala Lumpur

Avec quelque 2 millions d'habitants - agglomération comprise - Kuala Lumpur constitue le centre économique et politique du pays. Dotée du statut de territoire fédéral depuis 1974 et placée sous le contrôle du gouvernement central, elle est enclavée dans l'Etat de Selangor (côte occidentale de la péninsule). Créée vers 1860, Kuala Lumpur est une ville cosmopolite et animée, caractérisée par une grande variété de styles architecturaux : maisons traditionnelles, architecture coloniale, bâtiments ultra-modernes dont les deux tours Petronas, les plus hautes du monde (450 m). L'aéroport international Klia se trouve à 75 km du centre-ville (soit 28 mn par train rapide). Depuis février 2002, la capitale est la ville nouvelle de Putrajaya, centre administratif du gouvernement fédéral, à 25 km au sud de Kuala Lumpur (www.pjholds.com.my).

#### Autres villes
Sur la péninsule : Georgetown, capitale de l'Etat du Penang. Ipoh, capitale de l'Etat de Perak. Johor Bahru, ville la plus méridionale et capitale de l'Etat du Johor.
