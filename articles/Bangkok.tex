layout: blog
title: Bangkok
date: 2008/06/12 14:00:11
tags:
---

Alors alors... Je vous ai laissés à Hua Hin, petite ville en passe de devenir une grosse station balnéaire à 3 heures de bus de Bangkok. Je n'ai pas énormément de choses à vous dires à propos de cette ville, si ce n'est que le patron de la guest house "All Nations" est Anglais, et qu'il est fan de tennis. Nous avons regardé Roland Garros ensembles. Je suis resté 3 nuits a Hua Hin puis j'ai pris le bus pour Bangkok.

Pout tout vous dire j'ai hésité à appeler cet article "Marchés a Bangkok", ou "Marcher à Bangkok". C'est donc tout logiquement que j'ai choisi un troisième titre.

Bangkok... très grande ville, bien sympa à découvrir. On peut trouver tout ce que l'on veut il suffit de bien chercher. Cependant si vous voulez une montre Rolex ou une écharpe Gucci, le premier petit marchand ambulant fera l'affaire... Bien sûr n'allez pas lui demander un reçu ou un certificat de conformité.

J'ai été très bien accueilli par Cécile, une amie de mes parents qui est prof de maths à Bangkok, puis Célia, sa fille, nous a rejoint, elle rentrait d'un stage de 3 mois au Viêtnam. Autant vous dire qu'on a pas arrété de parler de voyages.

La visite de la ville a commencé en prenant un bateau sur un des klongs, les klongs sont des canaux, pas toujours accueillants par leur odeure, mais permettant de voir un coté non touristique. Une petite photo pour se mettre dans l'ambience, et la video après.



<div><object width="640" height="505"><param name="movie" value="http://www.dailymotion.com/swf/x5n17d&related=1"></param><param name="allowFullScreen" value="true"></param><param name="allowScriptAccess" value="always"></param><embed src="http://www.dailymotion.com/swf/x5n17d&related=1" type="application/x-shockwave-flash" width="640" height="505" allowFullScreen="true" allowScriptAccess="always"></embed></object></div>


Puis je suis allé voir le Jim Thomson museum. Il s'agit d'un petit musée regroupant la collection d'art d'un certain Jim Thomson comme vous l'aurez compris. J'ai surtout apprécié ça pour l'architecture des petites maisons en bois, on ne s'attend pas à voir ça en pleins centre de Bangkok.



Puis je me suis balladé dans la ville, je suis resté en tout un peu plus de 4 jours à Bangkok. Les Thaïlandais adorent leur roi, et ils le montrent. Le lundi est le jour du roi et il est impressionnant de voir le nombre de Thaïs qui s'habillent en jaune (sa couleur) ce jour la. Ils adorent aussi montrer leur drapeau accompagné du drapeau du roi (symbol royale sur fond jaune). Voici une photo d'un batiment prise.



Comme j'aime bien le bateau (vous le saviez?), je n'ai pas pu m'enpêcher de naviguer sur la Chao Praya, fleuve qui coule du côté Ouest de Bangkok, et c'est l'occasion de vous présenter des bonzes. Un bonze est un prêtre ou moine bouddhiste, très respecté de tous. J'ai fait attention à ne pas les faire tomber dans l'eau, ça aurait fait mauvais genre de couler un bonze dans la rivière (amis des mauvais jeux de mots, bonjour).



Un truc que j'adore maintenant, c'est visiter les marchés. Ici on voit tout, du plus propre (quoique...) au plus sale, mais tout le monde achète et finalement tout se passe bien. En fait c'est nous qui avons un sens du propre trop développé je pense. Cela dit la majorité des stands sont de tres bonne qualite, il faut juste fermer les yeux quand on passe devant certains stands de poisson ou de viande.





<div><object width="640" height="505"><param name="movie" value="http://www.dailymotion.com/swf/x5r4b7&related=1"></param><param name="allowFullScreen" value="true"></param><param name="allowScriptAccess" value="always"></param><embed src="http://www.dailymotion.com/swf/x5r4b7&related=1" type="application/x-shockwave-flash" width="640" height="505" allowFullScreen="true" allowScriptAccess="always"></embed></object></div>

Je ne vous dis pas au revoir ici, j'attaque directement l'article suivant sur Kanchanaburi.
