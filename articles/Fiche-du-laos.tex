layout: landscape
title: Fiche du Laos
date: 2008/06/26 12:18:42
tags:
---

### Drapeau



### Situation
#### Presentation générale

Avec une superficie de 236.800 km2, le Laos est enclavé au coeur de la péninsule Indochinoise. Il est bordé à l’Est et au Nord par le Viêtnam et la Chine, et limite à l’Ouest et au Sud par la Birmanie, la Thaïlande et le Cambodge.

Le Mekong traverse le Laos sur 1898 km et dessine le principal axe du pays.

Le point culminant du Laos, le Phou Bia avec 2820 m, se trouve au sud du plateau de Xieng-Khouang. Une quinzaine d’autres sommets dépassent 2000 m.

### Population

Au Laos, soixante-huit groupes ethniques sont officiellement recensés. Parmi les plus importants, il faut citer les Lao Lum, ou Lao des plaines, représentant 45 a 50% de la population, les Lao Theung, ou Lao des vallées, comptant pour 30% et les Lao Sung ou Lao des montagnes, dont les 15% se répartissent en Hmongs, Meos ou Yaos.

|                                       |       LAOS      |    FRANCE    |
| ------------------------------------- |: -------------: | -----------: |
|Population                             |     5,2 millions|  	59 millions|
|Densite                                |     22,8 hab/km²|   108 hab/km²|
|Accroissement naturel de la population |              2,8|          0,46|
|Esperance de vie                       |           59 ans|     78,86 ans|
|Urbanisation                           |            17 %|      75,6 ans|

### Climat

Le Laos est un pays au climat tropical humide, marqué par deux saisons :

- La saison sèche de septembre à avril, agréable en décembre janvier, très chaude en mars et avril : la température est comprise entre 14 degres et 25 degrés
- La saison des pluies d’avril a septembre, tres humide et tres chaude, avec des orages frequents : la temperature est comprise entre 30degres et 44 degrés

### Villes principales
#### Vientiane

Vientiane est la capitale politique et économique du Laos. Située sur la rive gauche du Mékong, elle compte environ 370.000 habitants. Elle devint pour la première fois capitale sous le regne du Roi Pothisarath (1520-1548). Elle fut rasée en 1826 par les Thaïs et retrouva son rang de capitale sous l’impulsion des Français au debut du XXème siècle.

#### Luang-Prabang

Luang-Prabang, avec 110.000 habitants, est l’ancienne capitale royale. Située au confluent de la Nam Khan et du Mékong, sur un site privilégié de la préhistoire, elle est la ville la plus riche du pays en pagodes.

#### Savannakhet

Forte de 147.000 habitants, Savannakhet est un port fluvial important, au carrefour d’un des postes frontières ouverts pour le commerce avec la Thaïlande et de la route menant au Viêtnam (Da-Nang).

#### Pakse

Avec 72.000 habitants, au carrefour des routes vers la Thaïlande et le Cambodge, cette ville est l’ancien fief de la famille de Champassak qui régnait sur le Sud.
