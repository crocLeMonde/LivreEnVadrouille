\section{Palawan}

19 déc. 2011

\begin{multicols}{2}

Salut tout le monde, voici quelques news des Philippines histoire de vous changer les idées et vous réchauffer un peu l'esprit.
Je commence par vous préciser que je néai pas du tout été concerné par la tempête qui a balayé le Sud du pays car elle est passée 2 jours après que je sois remonté vers Manille (au Nord, donc). Du coup pas d'inquiétude à avoir pour moi la météo est plutôt pas mal comme vous allez le voir.

Je suis donc passé le 8 décembre à Kuala Lumpur pour un transit forcé de 8 heures, j'en ai profité pour aller faire un tour dans la ville et je me suis retrouvé comme par hazard au même endroit que là où on avait été 4 ans plus tôt avec Patrick : les tours jumelles Petronas. Du coup j'en ai profité pour me refaire une série de photos de ces tours qui sont bien jolies architecturalement parlant je trouve.

%<img src="http://etienne.croclemonde.org/public/Philippines/DSCF2186.jpg" />

Puis c'est l'arrivée à Manille de nuit et sous la pluie, grande ville grouillante et sale par endroits, très clean dans d'autres, j'en parlerai plus longtemps dans le prochain article qui concernera le Nord du Pays. Pour l'heure reprenons la direction de l'aéroport pour prendre un vol vers Palawan où il y a parait il quelques belles plages à voir.

Arrivée à Puerto Princesa au centre de l'île, j'ai voulu marcher de l'aéroport à la ville et du coup je me suis fait doubler par tout le monde, plus de chambres de libres dans les endroits sympas. Je rencontre Jama, une américaine du Wyoming et on partage une chambre dans un truc un peu plus cher. Le lendemain matin départ pour le Nord de l'île, direction El Nido où je voulais au départ trouver un endroit où me poser un peu à l'écart de la ville pour être un peu tranquille. Ca c'était sans comter sur La Banane, petit hostal sur lequel je suis tombé, par et pour pour les backpackers avec une ambiance bien sympa. En fait je me suis laissé aller 5 jours les pieds dans l'eau avec d'autres personnes des US, du Canada, UK, Allemagne, Pays Bas, Suède, et j'en oubli.

Au programme du premier jour, location de motorbike pour faire une boucle de 80km dans tout le nord de l'île de Palawan. Le truc ici c'est que les routes n'en sont pas vraiment, c'est plutôt du style terre compressée par le temps soupoudrée de caillasses plus ou moins grosse, glissante ou pointues. Il a donc bien fallu faire gaffe à quelle bécane je prenait car le premier hopital sérieux est à 6h de route + 1h de vol (Manille quoi), autant dire qu'il n'y a aucune chance d'être soigné en moins de 24h en cas de pépin..

%<img src="http://etienne.croclemonde.org/public/Philippines/DSCF2193.jpg" />

J'aurai bien aimé avoir une planche de surf ou de body board sous la main..

%<img src="http://etienne.croclemonde.org/public/Philippines/DSCF2196.jpg" />

Puis petit tour à l'intérieur des terres pour faire des rencontres.

%<img src="http://etienne.croclemonde.org/public/Philippines/DSCF2198.jpg" />
%<img src="http://etienne.croclemonde.org/public/Philippines/DSCF2202.jpg" />
%<img src="http://etienne.croclemonde.org/public/Philippines/DSCF2218.jpg" />

Et hop, retour au bord de mer, dans un village de pêcheurs, super jolis les bateaux au bord de la plage..

%<img src="http://etienne.croclemonde.org/public/Philippines/DSCF2207.jpg" />
%<img src="http://etienne.croclemonde.org/public/Philippines/DSCF2210.jpg" />
%<img src="http://etienne.croclemonde.org/public/Philippines/DSCF2220.jpg" />

Au détour de la route, arrêt au bord d'un chantier pour essayer de discuter avec les ouvriers qui étaient en train de bétonner un petit canal avant de faire un pont au dessus. Malheureusement l'anglais n'etait pas leur tasse de thé, c'est con car l'ambiance était plutôt du genre décontractée ça aurait été sympa de se comprendre.

%<img src="http://etienne.croclemonde.org/public/Philippines/DSCF2222.jpg" />

Et là on attaque du lourd. dans les jours qui ont suivi ce tour en moto je suis allé avec le groupe de backpackers découvrir un peu les alentour, et je dois vous dire qu'on a pas été déçus !!

On attaque par une petite mise en bouche, juste comme ça pour donner le ton.

%<img src="http://etienne.croclemonde.org/public/Philippines/DSCF2224.jpg" />

Ah tiens ce serai pas l'heure du goûter ? Allez hop au lieu d'aller payer une fortune dans les terrasses des alentours pour boire un verre, ce serai quand même plus sympa de demander à un mec si il peut nous descendre une coco chacun, trois coups de machette plus tard on avait plus soif :)

%<img src="http://etienne.croclemonde.org/public/Philippines/DSCF2230.jpg" />

Quand je pense que certains vont s'entasser sur la cote d'Azur.. Palawan les mecs !!

%<img src="http://etienne.croclemonde.org/public/Philippines/DSCF2233.jpg" />
%<img src="http://etienne.croclemonde.org/public/Philippines/DSCF2236.jpg" />

Le lendemain, comme El Nido est réputé pour l'archipel d'île qu il y a en face, il était impossible que je néaille pas faire du snorkeling là bas. Bon le problème c'est que le seul moyen d'aller faire un tour de l'archipel est de participer à un charter de touristes sur une pirogue, et ça c'est pas trop mon truc. Du coup avec quelques autres Français et un Suédois on s'est monté notre propre tour en trouvant un guide et sa pirogue, le gros intérêt céest que du coup on est pas allés faire les circuits des autres et on ne séest pas retrouvés aux mêmes moments dans ces endroits magiques, nous étions donc les seuls sur les plages et dans les lagons, et ca, jéai bien apprécié :)

%<img src="http://etienne.croclemonde.org/public/Philippines/DSCF2256.jpg" />
%<img src="http://etienne.croclemonde.org/public/Philippines/DSCF2258.jpg" />
%<img src="http://etienne.croclemonde.org/public/Philippines/DSCF2261.jpg" />
%<img src="http://etienne.croclemonde.org/public/Philippines/DSCF2263.jpg" />
%<img src="http://etienne.croclemonde.org/public/Philippines/DSCF2264.jpg" />

Mais au fait, vous savez a quoi servent les archipels ?

\end{multicols}

\bigskip
\textbf{\textsc{Commentaires}}

\medskip
Tatid a écrit le 19 déc. 2011 :
\begin{displayquote}
Woooow encore des bien jolis photos et des endroits magnifiques ! Lucky you!

Bon m'sieur le geek, va par contre falloir thuner le flux RSS pour que le contenu du post s'affiche direct dans Google Reader ;-).
\end{displayquote}

\medskip
Dova a écrit le 21 déc. 2011 :
\begin{displayquote}
Rrrrrrrrggghhhhh, pendant ce temps là, nous, pauvres parisiens, sommes sous la pluie et la grisaille!
Profites bien de tout ce qui t'entoure et merci pour ces récits et très très belles photos.
La bise.
\end{displayquote}

\medskip
Dorian a écrit le 21 déc. 2011 :
\begin{displayquote}
Eh ouais, yen a qui ont de la chance ! La chaleur, les plages désertes paradisiaques...

Allez, profite bien et continue à nous faire bisquer un peu ;)
\end{displayquote}

\vfill

