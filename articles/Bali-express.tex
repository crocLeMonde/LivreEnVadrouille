\section{Bali express}

Date: 18/01/2012

\begin{multicols}{2}


La suite du voyage se passe donc en tendem jusqu'au Viêtnam, et nous l'écrivons donc à deux.

Après une première prise en main des scooters pour Florian, nous partons vers le sud à la découverte des plage de Nusa Dua, lieu privilégié pour les hôtels de luxe, le surf et la nature :

%<img src="http://etienne.croclemonde.org/public/indonesie/DSC_0011.jpg" />
%<img src="http://etienne.croclemonde.org/public/indonesie/DSC_0051.jpg" />

Dès le lendemain nous sommes vers le centre de Bali pour passer le nouvel an à Ubud, hors de la meute de touristes Australiens du sud de l'île. Au programme, visite de la ville et de boutiques de souvenirs sous la pluie... eh oui c'est les risques du metier quand on part en voyage durant la mousson.

%<img src="http://etienne.croclemonde.org/public/indonesie/DSC_0090.jpg" />

En soirée, visite d'un temple, resto et feux d'artifices pour entrer en 2012.

%<img src="http://etienne.croclemonde.org/public/indonesie/DSC_0117.jpg" />

Puis dans les jours qui ont suivi on est partis pour un tours de l'île, meilleur moyen pour observer de magnifiques rizieres. Sur la route de Tabanan :

%<img src="http://etienne.croclemonde.org/public/indonesie/DSC_0147.jpg" />
%<img src="http://etienne.croclemonde.org/public/indonesie/DSC_0184.jpg" />

Mais au fait, on a pas fait une photo de groupe ? Ca, c'est nous :

%<img src="http://etienne.croclemonde.org/public/indonesie/DSC_0153.jpg" />

A quelques heures de motorbike de la, quelque part sur la cote du nord de l'île, on passe par quelques villages de pêcheurs où les bateaux n'attendent que nous pour mettre les voiles.

%<img src="http://etienne.croclemonde.org/public/indonesie/DSC_0154.jpg" />

Après quelques kilomètres, juste avant d'arriver à notre destination, devinez qui on croise ? Le petit cousin de King Kong.

%<img src="http://etienne.croclemonde.org/public/indonesie/DSC_0166.jpg" />

Le lendemain nous sommes de nouveau sur les routes, pour aller voir Ugung Batur, le plus gros volcan de l'île, en passant par une multitude de petits temples typiques... c'est même un volcan dans un volcan.

%<img src="http://etienne.croclemonde.org/public/indonesie/DSC_0171.jpg" />
%<img src="http://etienne.croclemonde.org/public/indonesie/DSC_0183.jpg" />
%<img src="http://etienne.croclemonde.org/public/indonesie/DSC_0191.jpg" />

Puis le tour de l'île touche à sa fin, et comme je l'avais promis à Sales nous sommes repassés le voir. C'est l'occasion de vous montrer quelques photos des alentours de chez lui ainsi que du bungalow qu'il a construit pour accueillir des touristes, et son warung, magasin pour le village.

%<img src="http://etienne.croclemonde.org/public/indonesie/DSC_0209.jpg" />
%<img src="http://etienne.croclemonde.org/public/indonesie/DSC_0223.jpg" />

Et là, on en prend plein les yeux..

%<img src="http://etienne.croclemonde.org/public/indonesie/DSC_0232.jpg" />
%<img src="http://etienne.croclemonde.org/public/indonesie/DSC_0238.jpg" />
%<img src="http://etienne.croclemonde.org/public/indonesie/DSC_0253.jpg" />

C'est bientôt l'heure du départ. Sales tient à nous faire une recette typiquement balinaise, a base de poulet farci d'herbes aussi diverses que variées dont seul lui a le secret: LE chicken tutu.

Mais avant la degustation, il nous faut aller chercher les ingredients au marché, en commencant par le poulet... bien frais !

%<img src="http://etienne.croclemonde.org/public/indonesie/DSC_0213.jpg" />

Le poulet est farci puis mis en papillotte dans une feuille de bananier avant de mijoter pour plus de tendresse et un feu d'artifice de saveurs.

%<img src="http://etienne.croclemonde.org/public/indonesie/DSC_0235.jpg" />

Apres 5h de cuisson a feu tres doux, il ne reste plus qu'à dresser les assiettes: selamat makan bien sûr !

%<img src="http://etienne.croclemonde.org/public/indonesie/DSC_0257.jpg" />

Une fois le chicken tutu dégusté, il était temps de reprendre la route vers le sud pour redécoller vers Ho Chi Minh Ville.

To be continued...

\end{multicols}

\bigskip
\textbf{\textsc{Commentaires}}

\medskip
Salah  a écrit le 22 janv. 2012 :
\begin{displayquote}
Hello les mecs,

J'espère que vous vous amusez bien. Petit message pour vous dire mes voeux de bonheur pour cette année et surtout pour souhaiter un bon anniv à Etienne.

A+ et envoyez d'autres tofs.
\end{displayquote}

\medskip
Macolu a écrit le 22 janv. 2012 :
\begin{displayquote}
Ça donne faim tout ça !

Profitez-en bien, et bon anniversaire Étienne !
\end{displayquote}

\medskip
Jaco a écrit le 22 janv. 2012 :
\begin{displayquote}
Ouais j'ai une méga dalle là !
Bonne continuation et bon anniversaire ma poule !
\end{displayquote}

\medskip
Royal a écrit le 23 janv. 2012 :
\begin{displayquote}
Bon anniv' !

Enjoy !
\end{displayquote}

\medskip
Ségolène a écrit le 2 fév. 2012 :
\begin{displayquote}
Hello!!!!

C'est magnifique!!! tu nous donnes envie de te rejoindre!!!
En tout cas prépare toi au froid car je te garantis que le temps ici est différent mais alors si tu savais...
Le bonjour de Dorian.
\end{displayquote}

\medskip
Emmanuel Higel a écrit le 19 mars 2015 :
\begin{displayquote}
Bonjour à vous,

Je me permets de vous contacter pour vous présenter mon site www.tresorsdumonde.fr et pour vous proposer un partenariat par la même occasion.

Tresorsdumonde.fr est un site sur lequel je travaille depuis bientôt 7 mois. Le but du site est de présenter tous les jours à l'internaute un nouveau lieu dans le monde.
J'en suis pour le moment à un peu plus de 210 lieux référencés et je compte, sur le long terme, atteindre les 1000 lieux.
Le gros point fort de tresorsdumonde.fr est sa carte interactive avec la position de tous les lieux.
J'ai donc pour le moment 16 trésors en Chine et je ne compte pas m'arrêter en si bon chemin ;)

Pour en venir au partenariat, J'ai une page facebook de plus de 30 000 fans sur laquelle je vous propose de publier un ou deux posts à propos de votre site Internet:
https://www.facebook.com/LesTresorsDuMonde

Je peux autrement vous proposer un lien depuis mon blog de voyage personnel (www.emmanuelhigel.com) si vous avez un article à me communiquer.

En contrepartie, pourriez-vous me faire un lien vers tresorsdumonde.fr sur l'une de vos pages ou articles?

Je reste également ouvert à toutes autres propositions qui pourraient nous être bénéfique à tous les deux :)

Merci d'avance pour votre réponse,

Emmanuel.
\end{displayquote}

\vfill

