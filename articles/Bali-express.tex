layout: landscape
title: Bali express
date: 2012/01/18 16:39:42
tags:
---


La suite du voyage se passe donc en tendem jusqu'au Viêtnam, et nous l'écrivons donc à deux.

Après une première prise en main des scooters pour Florian, nous partons vers le sud à la découverte des plage de Nusa Dua, lieu privilégié pour les hôtels de luxe, le surf et la nature :

<img src="http://etienne.croclemonde.org/public/indonesie/DSC_0011.jpg" />
<img src="http://etienne.croclemonde.org/public/indonesie/DSC_0051.jpg" />

Dès le lendemain nous sommes vers le centre de Bali pour passer le nouvel an à Ubud, hors de la meute de touristes Australiens du sud de l'île. Au programme, visite de la ville et de boutiques de souvenirs sous la pluie... eh oui c'est les risques du metier quand on part en voyage durant la mousson.

<img src="http://etienne.croclemonde.org/public/indonesie/DSC_0090.jpg" />

En soirée, visite d'un temple, resto et feux d'artifices pour entrer en 2012.

<img src="http://etienne.croclemonde.org/public/indonesie/DSC_0117.jpg" />

Puis dans les jours qui ont suivi on est partis pour un tours de l'île, meilleur moyen pour observer de magnifiques rizieres. Sur la route de Tabanan :

<img src="http://etienne.croclemonde.org/public/indonesie/DSC_0147.jpg" />
<img src="http://etienne.croclemonde.org/public/indonesie/DSC_0184.jpg" />

Mais au fait, on a pas fait une photo de groupe ? Ca, c'est nous :

<img src="http://etienne.croclemonde.org/public/indonesie/DSC_0153.jpg" />

A quelques heures de motorbike de la, quelque part sur la cote du nord de l'île, on passe par quelques villages de pêcheurs où les bateaux n'attendent que nous pour mettre les voiles.

<img src="http://etienne.croclemonde.org/public/indonesie/DSC_0154.jpg" />

Après quelques kilomètres, juste avant d'arriver à notre destination, devinez qui on croise ? Le petit cousin de King Kong.

<img src="http://etienne.croclemonde.org/public/indonesie/DSC_0166.jpg" />

Le lendemain nous sommes de nouveau sur les routes, pour aller voir Ugung Batur, le plus gros volcan de l'île, en passant par une multitude de petits temples typiques... c'est même un volcan dans un volcan.

<img src="http://etienne.croclemonde.org/public/indonesie/DSC_0171.jpg" />
<img src="http://etienne.croclemonde.org/public/indonesie/DSC_0183.jpg" />
<img src="http://etienne.croclemonde.org/public/indonesie/DSC_0191.jpg" />

Puis le tour de l'île touche à sa fin, et comme je l'avais promis à Sales nous sommes repassés le voir. C'est l'occasion de vous montrer quelques photos des alentours de chez lui ainsi que du bungalow qu'il a construit pour accueillir des touristes, et son warung, magasin pour le village.

<img src="http://etienne.croclemonde.org/public/indonesie/DSC_0209.jpg" />
<img src="http://etienne.croclemonde.org/public/indonesie/DSC_0223.jpg" />

Et là, on en prend plein les yeux..

<img src="http://etienne.croclemonde.org/public/indonesie/DSC_0232.jpg" />
<img src="http://etienne.croclemonde.org/public/indonesie/DSC_0238.jpg" />
<img src="http://etienne.croclemonde.org/public/indonesie/DSC_0253.jpg" />

C'est bientôt l'heure du départ. Sales tient à nous faire une recette typiquement balinaise, a base de poulet farci d'herbes aussi diverses que variées dont seul lui a le secret: LE chicken tutu.

Mais avant la degustation, il nous faut aller chercher les ingredients au marché, en commencant par le poulet... bien frais !

<img src="http://etienne.croclemonde.org/public/indonesie/DSC_0213.jpg" />

Le poulet est farci puis mis en papillotte dans une feuille de bananier avant de mijoter pour plus de tendresse et un feu d'artifice de saveurs.

<img src="http://etienne.croclemonde.org/public/indonesie/DSC_0235.jpg" />

Apres 5h de cuisson a feu tres doux, il ne reste plus qu'à dresser les assiettes: selamat makan bien sûr !

<img src="http://etienne.croclemonde.org/public/indonesie/DSC_0257.jpg" />

Une fois le chicken tutu dégusté, il était temps de reprendre la route vers le sud pour redécoller vers Ho Chi Minh Ville.

To be continued...
