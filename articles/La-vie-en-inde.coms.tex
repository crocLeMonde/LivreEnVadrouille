\bigskip
\textbf{\textsc{Commentaires}}

\medskip
Peggy a écrit le 8 fév. 2008 :
\begin{displayquote}
C'est magique ton récit.
\end{displayquote}

\medskip
Titou a écrit le 8 fév. 2008 :
\begin{displayquote}
Toujours un grand plaisir de vous lire (au boulot) ! C'est vrai que ça a l'air d'être un univers complètement différent. J'ai beau être parti loin il n'y a rien de comparable. Faites gaffe à vous quand même avec les ti soucis sanitaires et surtout profitez toujours à fond de cette super expérience que vous êtes en train de vivre ! Merci de nous la faire partager. A très bientôt ! Biz à vous deux !
\end{displayquote}

\medskip
Poun's a écrit le 8 fév. 2008 :
\begin{displayquote}
C'est super de pouvoir partager un peu de votre expédition grâce aux commentaires et aux photos. Je pense qu'on est pas mal à attendre impatiemment le prochain épisode!
A bientôt.
\end{displayquote}

\medskip
Sam a écrit le 9 fév. 2008 :
\begin{displayquote}
Pfiou :) Je vous admire de mener une telle aventure ! Je ne pense pas pouvoir survivre à ça pendant plus de 48h.
En tout cas, vous voyez et vivez des choses magnifiques, c'est le principal ! Continuez à déjouer les arnaques et à vous en prendre plein la vue pour nous tous !
\end{displayquote}

\medskip
Les vadrouilleurs a écrit le 10 fév. 2008 :
\begin{displayquote}
Merci beaucoup pour vos messages, c'est vrai qu'on nous avait prévenu qu'on se prendrait une grosse claque en arrivant ici, et heureusement d'ailleurs car sinon\dots
\end{displayquote}

\medskip
Seb a écrit le 11 fév. 2008 :
\begin{displayquote}
Super sympa de nous faire partager tous ça, ça a l'air d'être super impréssionant là bas.
Bonne suite et que l'aventure continue.
\end{displayquote}

\medskip
Catherine a écrit le 11 fév. 2008 :
\begin{displayquote}
Vous me donnez de plus en plus envie de partir en Inde en juillet. Mais j'attends votre avis : est-ce profitable pour François et Marine ou faut-il attendre qu'ils soient plus agés ?
métier oblige : je suis sensible à la demande pour les écoliers et aimerais savoir ce qu'il est souhaitable d'envoyer et à quelle adresse ? (en recommandé où le facteur ne risque pas de se servir ?)
Vous nous transmettrez l'adresse à laquelle on vous enverra un peu d'argent quand les arnaques auront eu raison de votre porte-monnaie\dots
merci beaucoup à Cécile pour son appel.
\end{displayquote}

\medskip
La vadrouilleuse a écrit le 13 fév. 2008 :
\begin{displayquote}
Pour répondre au dernier commentaire, voici notre avis sur le fait d'emmener des enfants en vacances en Inde. On est catégoriques, d'apres notre expérience, non, les enfants restent à la maison. Trop de choses ici peuvent heurter leur sensibilité et les marquer à vie.
De même qu'une femme seule, ou même deux jeunes femmes, ou une femme et ses enfants seront ici, on pense, sans arrêt regardés et il vaut mieux s'abstenir de voyager dans ces conditions. La condition de la femme n'est pas la même ici qu'en Europe.
\end{displayquote}

\medskip
Tatid a écrit le 21 fév. 2008 :
\begin{displayquote}
Je le disais dans un commentaire précédent (ou suivant plutôt puisque je lis le blog dans le sens non-chronologique :p), niveau confort et culture, ça doit faire un méga choc, surtout pour nous occidentaux qui avons besoin d'une certaine hygiène quoi\dots C'est vrai que je serais méfiant quant à la nourriture que je pourrais manger\dots
Sinon, vue comment vous décrivez ce qui se passe dans la rue, ça doit être un sacré bordel :-D Quant aux arnaques, je trouve ça abusé, profiter des touristes pour leur faire cracher de la thune tout le temps, grrr\dots
Félicitation quand même d'avoir eu le courage de vivre cette aventure (qui en est vraiment une en soi) !
Grâce à ce voyage, d'après ce que vous racontez, vous avez rencontré des gens assez exceptionnels, je pensais à Raj là, c'est génial ce qu'il fait pour les autres, ça doit être touchant de voir ces enfants pris en charge par un "grand enfant" qui veut leur apporter de la connaissance\dots J'ai hâte de voir les photos !
\end{displayquote}

\vfill
