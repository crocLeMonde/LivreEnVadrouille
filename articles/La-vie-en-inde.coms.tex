\bigskip
\textbf{\textsc{Commentaires}}

\medskip
Peggy a écrit le 08 fév. 2008 :
\begin{displayquote}
C'est magique ton récit.
\end{displayquote}

\medskip
Titou a écrit le 08 fév. 2008 :
\begin{displayquote}
Toujours un grand plaisir de vous lire (au boulot) ! C'est vrai que ca a l'air d'etre un univers completement different. J'ai beau etre parti loin il n'y a rien de comparable. Faites gaffe a vous quand meme avec les ti soucis sanitaires et surtout profitez toujours a fond de cette super experience que vous etes en train de vivre ! Merci de nous la faire partager. A tres bientot ! Biz a vous deux !
\end{displayquote}

\medskip
Poun's a écrit le 08 fév. 2008 :
\begin{displayquote}
C'est super de pouvoir partager un peu de votre expédition grâce aux commentaires et aux photos. Je pense qu'on est pas mal à attendre impatiemment le prochain épisode!
A bientôt.
\end{displayquote}

\medskip
Sam a écrit le 09 fév. 2008 :
\begin{displayquote}
Pfiou :) Je vous admire de mener une telle aventure ! Je ne pense pas pouvoir survivre à ça pendant plus de 48h.
En tout cas, vous voyez et vivez des choses magnifiques, c'est le principal ! Continuez à déjouer les arnaques et à vous en prendre plein la vue pour nous tous !
\end{displayquote}

\medskip
Les vadrouilleurs a écrit le 10 fév. 2008 :
\begin{displayquote}
Merci beaucoup pour vos messages, c est vrai qu on nous avait prevenu qu on se prendrait une grosse claque en arrivant ici, et heureusement d ailleurs car sinon...
\end{displayquote}

\medskip
Seb a écrit le 11 fév. 2008 :
\begin{displayquote}
Super simpa de nous faire partagé tous ca , ca a l'air d'être super imprésionant là bas.
Bon suite et que l'aventure continue.
\end{displayquote}

\medskip
Catherine a écrit le 11 fév. 2008 :
\begin{displayquote}
Vous me donnez de plus en plus envie de partir en Inde en juillet. Mais j'attends votre avis :est-ce profitable pour François et Marine ou faut-il attendre qu'ils soient plus agés ?
        métier oblige: je suis sensible à la demande pour les écoliers et aimerais savoir ce qu'il est souhaitable d'envoyer et à quelle adresse ? (en recommandé ou le facteur ne risque pas de se servir ?)
        Vous nous transmettrez l'adresse à laquelle on vous enverra un peu d'argent quand les arnaques auront eu raison de votre porte-monnaie ...
                merci beaucoup à Cécile pour son appel.
\end{displayquote}

\medskip
La vadrouilleuse a écrit le 13 fév. 2008 :
\begin{displayquote}
Pour repondre au dernier commentaire, voici notre avis sur le fait d emmener des enfants en vacances en Inde. On est categoriques, d apres notre experience, non, les enfants restent a la maison. Trop de choses ici peuvent heurter leur sensibilite et les marquer a vie.
De meme qu une femme seule, ou meme deux jeunes femmes, ou une femme et ses enfants seront ici, on pense, sans arret regardees et il vaut mieux s abstenir de voyager dans ces conditions. La condition de la femme n est pas la meme ici qu en europe.
\end{displayquote}

\medskip
Tatid a écrit le 21 fév. 2008 :
\begin{displayquote}
Je le disais dans un commentaire précédent (ou suivant plutôt puisque je lis le blog dans le sens non-chronologique :p), niveau confort et culture, ça doit faire un méga choc, surtout pour nous occidentaux qui avont besoin d'une certaine hygiène quoi... C'est vrai que je serais méfiant quant à la nourriture que je pourrais manger...
Sinon, vue comment vous décrivez ce qui se passe dans la rue, ça doit être un sacré bordel :-D Quant aux arnaques, je trouve ça abusé, profiter des touristes pour leur faire cracher de la thune tout le temps, grrr...
Félicitation quand même d'avoir eu le courage de vivre cette aventure (qui en est vraiment une en soi) !
Grâce à ce voyage, d'après ce que vous racontez, vous avez rencontré des gens assez exceptionnels, je pensais à Raj là, c'est génial ce qu'il fait pour les autres, ça doit être touchant de voir ces enfants pris en charge par un "grand enfant" qui veut leur apporter de la connaissance... J'ai hâte de voir les photos !
\end{displayquote}

