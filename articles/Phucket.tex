layout: landscape
title: Phuket
date: 2008/05/30 14:58:25
tags:
---

Salut tout le monde...

Comment ça je me fais attendre, je vois vraiment pas de quoi vous voulez parler... Mais l'avantage quand les articles sont éspacés dans le temps c'est qu'à priori il va y en avoir plus à lire !! Premièrement j'ai ajouté une video (au debut) et 3 photos (à la fin) sur l'article de Langkawi.

Nous voici donc repartis en mer de Langkawi direction Phucket en quelques jours, avec haltes sur différentes îles se trouvant sur notre chemin. But avoué : faire de la plongée car la mer devenait enfin claire. Depuis l'Indonésie impossible de plonger, à certains endroit on ne voyait à peine le bout de notre bras dans l'eau. Tous ça c'est bien joli, mais c'était compter sans les nuages et la pluie qui rendent la plongée tout de suite moins agréable. Enfin bon on va pas se pleindre quand meme.

Nous passons près de Koh Lanta, ça vous dit quelque chose ?

<div><object width="640" height="505"><param name="movie" value="http://www.dailymotion.com/swf/x5lrzm&related=1"></param><param name="allowFullScreen" value="true"></param><param name="allowScriptAccess" value="always"></param><embed src="http://www.dailymotion.com/swf/x5lrzm&related=1" type="application/x-shockwave-flash" width="640" height="505" allowFullScreen="true" allowScriptAccess="always"></embed></object></div>

Puis nous voici arrivés à Ko Phi Phi, île très touristique, bien que nous soyons en basse saison, le trop plein de monde est blasant. Voici quand meme une belle photo, c'est le plus beau coté de la plage principale, l'autres coté étant occupé par des hôtels, restos, boutiques à souvenir...



Charlotte, tu reconnais ?

Mais Ko Phi Phi c'est pas pour le tourisme que vous aviez peut être déjà ce nom dans la tête, c'est un des endroits qui a été le plus touché par le tsunami en 2004 et du coup maintenant on se retrouve avec pleins de panneaux de partout indiquant où aller en cas d'alerte.



Vous remarquerez au passage l'écriture thaï, non non ce n'est pas de la décoration pour faire joli, c'est bien du texte... En plus de ça pour une même orthographe, un mot peut avoir jusqu'à 5 significations differentes suivant sa prononciation. Aprenez le thaï, qu'il disait !!

Ca y est, on arrive a Phuket. Et la, question tourisme, on est en plein dedans : des plages, du béton et... de la prostitution. Phuket Town, la plus grosse ville de l'île échape un peu à ça, c'est sympa de s'y ballader, on retrouve le quartier chinois, le marché couvert...

<div><object width="640" height="505"><param name="movie" value="http://www.dailymotion.com/swf/x5lrj3&related=1"></param><param name="allowFullScreen" value="true"></param><param name="allowScriptAccess" value="always"></param><embed src="http://www.dailymotion.com/swf/x5lrj3&related=1" type="application/x-shockwave-flash" width="640" height="505" allowFullScreen="true" allowScriptAccess="always"></embed></object></div>

Sur le reste de l'île, des plages...



... et Boat Lagoon, la marina où nous nous sommes arrêtés pour sortir le bateau. En effet Patrick a besoin de faire un carrénage sur le bateau (remettre de la peinture spéciale anti coquillages appelée anti fulling sur la coque) et faire quelques réparations.





Bon... et j'avais pas dit qu'une fois arrivé a Phuket je reprenais mon sac, moi? C'est partiiiiiiii !!!

Jeudi matin, mon sac est fait, je quitte Patrick et son bateau perché à 4 mètres de haut direction Phuket Town. Il faut que j'achète le Lonely Planet car je n'ai pas de guide de la Thaïlande, et que je prenne le bus direction Hua Hin, entre Phuket et Bangkok. Première librairie : pas de guide. Deuxième : les Lonely Planet sont au rayon fiction, ca promet... et ils n'ont pas la Thaïlande, c'est vrai que c'est quand même beaucoup plus utile d'avoir le Brésil ou l'Australie à Phuket. Et puis ils sont aimables comme des portes de prisons, j'arrête, je vais prendre le bus.

Hua Hin please. Je paie, je monte dans le bus, il est 12h30, je n'ai aucune idée du temps de trajet. En premier on me dit 5h, puis 7, puis... j'arrive à 1h du matin apràs plus de 12 heures de trajet. Je marche un peu, refuse les désormais classiques taxi qui proposent des "cheap hôtel", et zou, je trouve une petite guest house pour dormir, la seule du coin à être ouverte 24h/24 à priori. Bon en même temps je stressais pas trop car Hua Hin est située le long de la mer, et si je ne trouvais rien je serais allé dormir sur la plage, il fait chaud ici y a pas de soucis.

Voila, je crois que je vous ai donné un peu de lecture, à bientôt.
